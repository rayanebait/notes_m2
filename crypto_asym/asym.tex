\documentclass[12pt]{article}
\usepackage[dvipsnames]{xcolor}
\usepackage{hyperref, pagecolor, mdframed }
\usepackage{tabularx, graphicx, amsmath, latexsym, amsfonts, amssymb, amsthm,
amscd, geometry, xspace, enumerate, mathtools}
\usepackage{tikz}

\theoremstyle{plain}
\newtheorem{thm}[subsubsection]{Th\'eor\`eme}
\newtheorem{lem}[subsubsection]{Lemme}
\newtheorem{cor}[subsubsection]{Corollaire}

\theoremstyle{definition}
\newtheorem{defn}[subsubsection]{Definition}
\newtheorem{prop}[subsubsection]{Proposition}

\newcommand{\fdiv}{\textrm{div}}
\newcommand{\Z}{\mathbb{Z}}
\newcommand{\N}{\mathbb{N}}
\newcommand{\Q}{\mathbb{Q}}
\newcommand{\F}{\mathbb{F}}
\newcommand{\algK}{\overline{K}}
\newcommand{\algF}{\overline{\mathbb{F}}}
\newcommand{\Pic}{\textrm{Pic}}
\newcommand{\Hom}{\textrm{Hom}}
\newcommand{\End}{\textrm{End}}
\newcommand{\Disc}{\textrm{Disc}}
\newcommand{\Det}{\textrm{Det}}
\newcommand{\Tr}{\textrm{Tr}}
\newcommand{\Or}{\mathcal{O}}
\newcommand{\OK}{\mathcal{O}_{K}}
\newcommand{\OL}{\mathcal{O}_{L}}
\newcommand{\C}{\mathbb{C}}
\newcommand{\ai}{\mathfrak{a}}
\newcommand{\bi}{\mathfrak{b}}
\newcommand{\w}{\omega}
\newcommand{\gr}{\color{Sepia}}
\newcommand{\rg}{\color{Red}}
\hypersetup{
    colorlinks=true,
    linkcolor=blue,
    urlcolor=Green,
    filecolor=RoyalPurple
}

\newcolumntype{M}[1]{>{\raggedright}m{#1}}
\definecolor{wgrey}{RGB}{148, 38, 55}


\begin{document}
\title{Cryptographie asymétrique}
\date{19 septembre 2023}
\maketitle

\section{Intro}
L'asymétrique ne sert pas à chiffrer mais plutot aux echanges de clés, etc..\\
\textbf{Ansi} recommande 
\begin{itemize}
    \item Des clés de 80 à 100 bits pour un niveau moyen de sécu (données ne durant pas dans le temps $\sim$ minutes)
    \item $>$ 100 bits : forts
\end{itemize}

\section{Arithmétique entiers}
La complexité est calc en fonction de :
\begin{itemize}
    \item La taille des données.
    \item ex : un entier $n$ en représentation binaire est en $log_2(n)=log(n)$.
\end{itemize}

A regarder : table de soustraction binaire lol.\\

\subsection{multiplication}
\begin{align*}
    && 11101=a&&\\
    &&\times 1101=b
\end{align*}

multiplication naive :
\begin{itemize}
    \item Taille(b) additions d'elts de taille a.
    \item Complexité : Taille(a)*Taille(b)
    \item Memoire : Taille(a*b)=Taille(a)+Taille(b)
\end{itemize}

Méthode de Karatsuba : $a,b\in \N$ et $k=log(a)=log(b)$. $a=\alpha2^{k/2}+\beta$, $b=\gamma2^{k/2}+\delta$. On écrit :
$$ab=\alpha\gamma2^k+(\alpha\gamma+\beta\delta-(\alpha-\beta)(\gamma-\delta))2^{k/2}+\beta\delta$$
On remarque que ya 3 multiplication d'élts de taille $k/2$ et $6$ soustr/add de taille $k/2$.\\

\begin{itemize}
    \item Complexité : $T(k)$ est donnée par 
    \begin{align*}
        3T(k/2)+6O(k/2)&=3^T(k/4)+6*3O(k/4)+6O(k/2)\\
        &=3^{log(k)}+2ck\sum_{i=1}^{log(k)}(3/2)^i\\
        &=3^{log(k)}+2Ck\frac{(3/2)^{log(k)} - 1}{(3/2) - 1}\\
        &=\dots\\
        &=O(k^{log(3)})
    \end{align*}

\end{itemize}

\subsection{division}
Division naive (euclidienne) :
\begin{itemize}
    \item Taille(a)-Taille(b)+1 soustraction de taille Taille(b).
    \item Complexité : O((taille(a)taille(b)+1)taille(b)).
    \item Mémoire : Taille(a)-Taille(b)+1  +  taille(b).
\end{itemize}


\subsection{algorithme d'euclide normal/etendu}
\begin{lem}
    Avec $a=r_0$, $b=r_1$, $r_{i}=q_{i+2}r_{i+1}+r_{i+2}$. On a $r_{i+2}<r_i/2$. Sauf pour les derniers i.
\end{lem}

D'ou
\begin{itemize}
    \item Au plus $log(a)$ divisions : i.e. $\sum_{i=0}^{k-1}(log(r_i)-log(r_{i+1}+1)log(r_i))\leq log(a)(k+log(a))$
    \item Complexité en $log(a)^2$
\end{itemize}

Euclide étendu : $u_0=1$, $u_1=0$ et $v_0=0$, $v_1=1$ et on écrit
\begin{align*}
    &u_{i+2}=u_i-q_iu_{i+1}\\
    &v_{i+2}=v_i-q_iv_{i+1}\\
\end{align*}

Pour calculer le pgcd :
\begin{itemize}
    \item Complexité : $O(log^2(a))$. (exo)
\end{itemize}

A montrer :
\begin{lem}
    $n$ un entier, calcul de la racine carrée entière de $n$ en $$O(log^3n)$$
\end{lem}

\subsection{indicatrice d'euler/inversion}
\begin{prop}
    $a^{-1}~mod~n$ se calcule en $$O(log^2(n))$$ grace a euclide
\end{prop}

\begin{defn}
    $\phi~:~\Z/n\Z\rightarrow\#\{0<i\leq n\}^*$
\end{defn}

\begin{prop}
    On veut $\phi(1)=1$ pour la récursion.
\end{prop}

\begin{prop}
    $\sum_{d\mid n}\phi(d)=n$
\end{prop}
Ca se prouve en posant $sum_{d\mid n}\phi(d)=f(n)$ alors : $$f(mn)=\sum_{d\mid mn}\phi(d)=\sum_{d_1\mid n}\sum_{d_2\mid m}\phi(d_1d_2)=f(m)f(n)$$. 
On écrit ducoup $f(n)=f(\prod p_i^{\alpha_i})$ et $f(p^{\alpha})=\sum_{k<\alpha} \phi(p^k)=\sum_{k}p^{k}-p^{k-1}=p^{\alpha}$

\begin{prop}
    $p\ne q$ deux nombres premiers et $n=pq$. On retrouve $p,q$ en $O(log^3(n))$ avec $n,\phi(n)$.
\end{prop}

\newpage
\section{corps finis}
$q=p^d$
\begin{prop}
    \begin{itemize}
        \item Complexité de l'addition/soustraction dans $\mathbb{F}_q$ : $O(log(q))$
        \item Complexité de la mult/l'inverse dans $\mathbb{F}_q$ : $O(log^2(q))$
    \end{itemize}
\end{prop}

Pour la multiplication : $2d-2$ calculs des sommes $\sum a_ib_{j-i}$ et $d$ mults a chaque fois puis $d$ additions. A la fin 
$O(log^2(q))$. 


\begin{prop}
    $d=gcd(n, q-1)$ racines n-emes de l'unité dans $\F_q$. $\F_q$ admet une racine primitive ssi $n\mid q-1$.
\end{prop}
Pour le deuxieme truc $(g^j)^n=1$ ssi $q-1\mid nj$ d'ou $q-1/d\mid j$ et on a $d$ valeurs possibles pour $j$.

\subsection{résidus quadratiques}
On prend $p\ne 2$ : 
\begin{prop}
    $x\mapsto \left(x^{p-1/2}\right)$ donne l'indice de $\F_p^{*2}$ et deux non résidus sont des puissances impaires donc le produit est une puissance paire.
\end{prop}

\begin{prop}
    C'est un morphisme de groupe.
\end{prop}

Maintenant on remplace $x\mapsto x^{p-1/2}$ par l'unique caractère abélien dans $\{\pm\}$ (Jacobi).
\subsection{Calcul de racine carrée, algo de shanks tonelli}
On réduit ca à un calcul de racine $2^{\alpha}$-eme de l'unité !

\begin{enumerate}
    \item On écrit $p-1=2^{\alpha}*s$, s impair.
    \item $r=a^{(s+1)/2}$
    \item on résoud $x^2a^{-1}\equiv 1~mod~p$
    \item En gros : $1\equiv a^{(p-1)/2}\equiv a^{2^{\alpha-1}s}\equiv (r^2a^{-1})^{2^{\alpha-1}}~mod~p$
    \item D'ou on cherche une racine de l'unité, $z$, alors $z^2\equiv y$ avec $y=r^2a^{-1}$.
    \item $z^2y\equiv y^{2^{\alpha -1}}~mod~p$ d'ou $(z^2y^{1-2^{\alpha-1}})^{2^{\alpha-1}}\equiv z^{2^{\alpha}}(y^{2^{\alpha-1}})^{1-2^{\alpha-1}}\equiv z^{2^{\alpha}}\equiv1~mod~p$
    \item D'ou il faut trouver une racine $2^{\alpha}$-eme de l'unité.
\end{enumerate}
Determination de la racine $2^{\alpha}$-eme de l'unité :
\begin{enumerate}
    \item Pour $\left(\frac{n}{p}\right)=-1$ on pose $b=n^s$
    \item Alors $\lvert b\rvert^{2^{\alpha}}$.
\end{enumerate}

On cherche ensuite le $b^j$ tel que $b^{2j}r^2a^{-1}\equiv1~mod~p$, on écrit $j=j_0+2j_1+...+2^{\alpha-1}j_{\alpha-1}$ :
\begin{enumerate}
    \item $b^{2j}r^2a^{-1}\equiv b^{2j_0+...+2^{\alpha}j_{\alpha-1}}\equiv b^{2j_0+...+2^{\alpha-1}j_{\alpha-2}}~mod~p$
    \item On regarde $(b^{2j}r^2a^{-1})^{2^{\alpha-2}}\equiv (b^{2^{\alpha-1}})^{j_0}a^{2^{\alpha-2}s}~mod~p$
    \item Comme $b^{2^{\alpha-1}}\equiv n^{(p-1)/2}\equiv -1~mod~p$
    \item Alors pour avoir $(b^{2j}r^2a^{-1})^{2^{\alpha-2}}\equiv 1$ il faut prendre $j_0=0$ ssi $(r^2a^{-1})^{2^{\alpha-1}}$
\end{enumerate}
Maintenant pour les autres coeffs que $j_0$, on suppose qu'on connait les $l<\alpha-2$ premiers tq 
$((b^{j_0+...+2^{l}j_l})r^2a^{-1})^{2^{\alpha-2-l}}~mod~p$ on cherche $j_{l+1}$ tq :
\begin{enumerate}
    \item $((b^{j_0+...+2^{l}j_l})r^2a^{-1})^{2^{\alpha-2-l}}~mod~p$
    \item On a $(b^j)^{2^{\alpha-2-l}}(r^2a^{-1})^{2^{\alpha-2-l-1}}\equiv b^{2^{\alpha-2-l}(j_0+2j_1+...+2^lj_l)}b^{2{\alpha-1}j_{l+1}}b^{2^{\alpha}(...)}...~mod~p$
    \item A nouveau on a $b^{2^{\alpha-1}j_{l+1}}\equiv (-1)^{j_{l+1}}$
    \item Et donc on pose $j_{l+1}=0$ ssi $((b^{j_0+...+2^lj_l})^2r^2a^{-1})^{2^{\alpha-2-l-1}}\equiv 1~mod~p$
\end{enumerate}


\section{Protocoles de cryptographie à clef publique}
Basé sur le principe de Kerkhoff.\\
\newline
\begin{tabular}{M{8 cm}  M{8 cm}}
    \textbf{Crypto symétrique} & \textbf{Crypto asymétrique}\\
    \tabularnewline
    $+$ rapide & $+$lent\\
    \tabularnewline
    1 clef partagée & $2$ clefs\\
    \tabularnewline
    $\times$ & Mise en reseau facile\\
    \tabularnewline
    Taille de clef petite & Taille de clé grande \\
\end{tabular}

Probleme de la crypto sym : nombre quadratique de clé par rapport au nb de personnes face a linéaire pour l'asym. ($+$ faut pouvoir échanger les clés)\\

\noindent Cryptographie asymétrique : \begin{enumerate}
    \item Authentification
    \item Echange de clefs
    \item Signature
\end{enumerate}


Etant donné une fct de chiffrement asym $f$ :
\begin{itemize}
    \item $f(m,k_{pub})=c$
    \item $f^{-1}(c, k_{priv})=m$
\end{itemize}
Authentification par challenge :
\begin{itemize}
    \item $f(challenge,k_{pub})\rightarrow c$ un challenge est donné et doit être dechiffré
    \item $challenge=m\leftarrow f^{-1}(c,k_{priv})$ 
\end{itemize}

\noindent Echange de clefs:
\begin{itemize}
    \item k la clef de session qu'on veut partager
    \item $f(k,k_{pub})\rightarrow c$ 
    \item $k=f^{-1}(c, k_{priv})$
\end{itemize}
\noindent Signature d'un message :
\begin{itemize}
    \item $f^{-1}(m, k_{priv})=sign$
    \item $f(sign, k_{pub})=m$
\end{itemize}

Propriétés d'une signature :
\begin{enumerate}
    \item Non-répudiable (irrévocable, on peut pas dire qu'on l'a pas signé)
    \item Le message est non-modifiable : inaltérable
    \item Authentique
    \item Non-réutilisable
    \item Infalsifiable
\end{enumerate}

\subsection{RSA}
Décrit \href{https://en.wikipedia.org/wiki/RSA_(cryptosystem)}{ici}. On
regarde des attaques sur RSA, les $p,q$ doivent être tous achetés ! 
\begin{defn}
    Attaque par module commun 
\end{defn}
Etant donné une communauté de $k$ personnes ayant tous.tes $p*q=n$. 
    Chaque utilisateurs recoit $(N, e_i(publique), d_i(privee))$. Si on connait $e_i,d_i$ alors 
    on sait que $e_id_i\equiv 1~mod~\phi(n)$ d'ou $e_id_i=1+k\phi(n)$. \\
    On pose $m=e_id_i-1=k\phi(n)$ d'ou $\forall a\in (\Z/N\Z)^{\times}$,$$a^m\equiv 1~mod~\phi(n)$$
    Or $4\mid \phi(n)$ donc $4\mid m$. \\
    Donc/etant donné $$a^m\equiv 1~mod~n$$.
    On a $a^{m/2}$ est une racine carrée de $1$ mod $n$(y'en a $4$). Si $a^{m/2}\equiv\alpha\neq \pm 1~mod~n$ 
    alors $(\alpha-1)(\alpha+1)\equiv 0~mod~N$ et $gcd(\alpha-1, n)\neq 1$ et 
    $gcd(\alpha-1, n)=p$. Soit $a\in (\Z/n\Z)^{\times}$. On pose : $m=2^ts$
    \begin{itemize}
        \item On calc $a^s~mod~n$, si $=\pm 1~mod~n$ on change $a$.
        \item Sinon on calc successivement $a^{2^is}~mod~n$. Et on s'arrete
        des qu'on trouve $1$.
        \item Si a l'étape d'avant on change $a$.
        \item sinon on a trouvé $\alpha$.
    \end{itemize}
Autre attaque : Si on chiffre $m$ pour deux destinataire : 
\begin{itemize}
    \item $c_1\equiv r^{e_1}~mod~n$
    \item $c_2\equiv r^{e_2}~mod~n$
\end{itemize}
Si $gcd(e_1,e_2)=1$ alors $\exists u,v\in\Z$ tq $ue_1+ve_2=1$. Donc
$c_1^u*c_2^v=m~mod~n$.\\

\begin{itemize}
    \item Besoin d'une fonction de hachage pour la signature
\end{itemize}

Alice ne veut pas signer $m$, Marvin choisit $r\in(\Z/n\Z)^{\times}$ et calcule
$m'=m*r^e~mod~n$. Alice signe $m'$, donc Marvin obtient $sign(m')\equiv m'^d\equiv (mr^e)\equiv m^dr$.\\

Nouvelle attaque 
\begin{defn}
    par exposant publique petit :
\end{defn}
On propose que tout le monde utilise le même e petit pour accélerer le chiffrement: 
$m$ est chiffré par $k$ utilisateurs differents:
$
\begin{cases}
    c_1\equiv m^e &mod~n_1\\
    \vdots&\\
    c_k\equiv m^e &mod~n_k
\end{cases}
$
Soit les $n_i$ sont premiers entre eux et on fait un lemme chinois, si $e<k$, $m^e<\prod_i n_i$.
Si pas premiers entre eux : gros pb.\\
\begin{defn}
    Attaque par petit exposant privé, but : améliorer la vitesse de déchiffrement.
\end{defn}
\begin{thm}
    Soit $N=pq$ avec $q<p<2q$ et $d=1/3\sqrt[4]{n}$. Etant donné le couple $(n,e)$ avec $ed\equiv1~mod~\phi(n)$, on peut
    retrouver efficacement $d$.
\end{thm}
\noindent\textbf{Preuve :} On pose $ed-k\phi(n)=1$. D'ou $\frac{e}{\phi(n)}-\frac{k}{d}=\frac{1}{d\phi(n)}$.
On approche $\phi(n)$ par $n$ et en utilisant le fait $d<1/3\sqrt[4]{n}$ on a :
$$\lvert \frac{e}{n}-\frac{k}{d}\rvert<\frac{1}{2d^2}$$
En passant par un dév en fractions continues à la bonne précision on retrouve $k/d$.\qed

RSA est pas indistinguable.(exponentiation binaire est rapide)

\subsection{Probleme de log discret}
Securité dépend du groupe dans lequel on travaille : Si on prend $G=(\Z/p\Z,+)$ et $h=gx~mod~p$ alors $x=hg^{-1}$, une étape.
\begin{defn}
    Problème de Diffie-Hellman(DHP): Etant donnés $g,g^a,g^b$ peut-on trouver $g^{ab}$.
\end{defn}

\begin{defn}
    Signature d'El Gamal : $k$ doit être secret et d'usage unique.
\end{defn}
\begin{itemize}
    \item $k$ doit être secret : a faire
    \item $k$ doit être d'usage unique : pareil
\end{itemize}


\newpage
\section{IGC(infrastucture de gestion de clefs)}

Gère les distributions de certificats.

\subsection{Création d'un certificat num}
2 modes:
\begin{itemize}
    \item Decentralisé: l'utilisateur crée son bi-clef
    \item Centralisé: L'autorité de confiance crée le bi-clef
\end{itemize}

\subsection{Révocation d'un certificat num}
\begin{itemize}
    \item Fin de limite de validité
    \item Révoque avant la date limite(par exemple 
    si oubli du mot de passe)
\end{itemize}

\subsection{Types de certificats}
\begin{itemize}
    \item Signature
    \item Chiffrement et signature
\end{itemize}
\subsection{Reste}
L'IGC est composée de $4$ entités obligatoires:
\begin{itemize}
    \item Autorité de confiance: signe les certifs
    \item Autorité d'enregistrement:s'assure de l'identité
    du demandeur de certifs.
    \item Autorité de dépôt: stocke les certifs et liste
    de révocations
    \item Entité finale: celle qui demande le certif
\end{itemize}
Optionnellement: Autorité de sequestre, conserve les clefs privées

\newpage
\section{Attaque sur log discret}

Dans un groupe cyclique $G=<g>$. On cherche 
étant donnés $h:=g^x$ à trouver $x$.

\begin{thm}[Shoup, 1997]
    Dans un groupe générique d'ordre \(p\) premier, 
    le calcul d'un log discret est au minimum en 
    $\Or(\sqrt{p})$.
\end{thm}

\subsection{Baby step, Giant step!}
\textbf{Idée:} Déterminer $s$ et $i$ tq $x=st+i$ où $t$
est choisi et $0\leq i<t$, i.e. \[h*(g^{-t})^s=g^i\]

\textbf{Pré-calcul:($\Or(t)$ étapes)} Calcul de tous les
 $g^i$ pour $0\leq i\leq t-1$ et $g^{-t}$.  (plus table
  de hashage des $g^i$)\\

\textbf{Calcul:($\Or(p/t)$ étapes)} Pour $s$ allant de
$0$ à $n/t$, on calcule $h*(g^{-t})^s$ et on teste
si $\exists i$ tq $h*(g^{-1})^s=g^i$. \qed
\newline

On prend $t=\sqrt{p}$ et on a la borne de Shoup.\\

\subsection{Algorithme Rho-Pollard}
\textbf{Idée: }Balade aléatoire dans $G$ en attendant
 une collision (d'ou la lettre $\rho$).\\

Il nous faut une fonction pseudo-aléatoire tq si 
$\alpha_i$ et $\beta_i$ sont tq 
    \[g_i=g^{\alpha_i} h^{\beta_i}\]
On puisse déterminer $\alpha_{i+1},\beta_{i+1}$ tq 
$F(g_i)=g_{i+1}=g^{\alpha_{i+1}}h^{\beta_{i+1}}$. 
Si on a une collision i.e. $g_i=g_j$ alors 
    \[g^{\alpha_i+x\beta_i}=g^{\alpha_j+x\beta_j}\]

\noindent D'ou $\alpha_i+x\beta_i\equiv \alpha_j+x\beta_j \mod n=
\lvert g\rvert$. Donc si $pgcd((\beta_i-\beta_j),n)=1$
alors \[x\equiv (\alpha_j-\alpha_i)(\beta_i-\beta_j)^{-1}
\mod n\].

\noindent \textbf{Algorithme:} On calcule la suite 
$(\alpha_i,\beta_i)$ avec $F$ et a chaque étape 
on regarde si y'a une collision.\\

\newpage
\noindent \textbf{Complexité:} Petite digression sur le 
\begin{center}
    Paradoxe des Anniversaires.
\end{center}
On regarde un tirage aléatoire uniforme sur $n$ boules 
avec remise. On cherche $P(n,k)$ la probabilité 
qu'après $k$-tirages on ait tiré au moins deux fois une 
meme boule: On regarde \[1-P(n,k)=\frac{n(n-1)\ldots
(n-(k-1))}{n^k}=\prod_{i=1}^{k-1}{1-i/n} \]

\noindent Or pour tout $x$ réel, $1+x\leq e^x$. D'ou 
    \[1-P(n,k)\leq \prod e^{-i/n}=e^{\sum (-i/n)}
    =e^{k(k-1)/2n}\]
On cherche quand $1-P(n,k)\geq 1/2$:
    \[e^{-(k(k-1))/2n}\leq 1/2\]

\noindent Donne $k(k-1)/2n\geq \ln(2)$, d'ou $k\geq
\sqrt{n}\sqrt{2 \ln(2)}$. Donc si on prend ce $k$ on a 
    \[P(n,k)\geq 1/2\]
(wow!).\\

\noindent \textbf{Retour à la complexité:} $\Or(\sqrt{n})$
en temps mais $\Or(\sqrt{n})$-mémoire..  Mais il existe 
une version sans mémoire de même complexité!

\subsection{Réduction de Pohlig-Hellman}
\textbf{Idée:} Au lieu de calculer un unique log discret 
dans un grand groupe on calcule plusieurs logs discrets 
dans des groupes plus petits. On suppose que $n=\prod p_i
^{\alpha_i}$ est composé $\rightarrow$ lemme chinois.
Plusieurs paramètres:
\begin{center}
    On pose $n_i=\frac{n}{p_i^{\alpha_i}}$, $g_i=g^{n_i}$ alors:
    \[\lvert g_i\rvert = p_i^{\alpha_i}\]
    Puis $h_i=h^{n_i}=(g^x)^{n_i}=g_i^x$
\end{center}

\noindent Et $x$ est determiné $\mod p_i^{\alpha_i}$. \\

\noindent \textbf{Algorithme mod $p^\alpha$:} 
On pose $x=\sum_{i=0}^{\alpha_i-1} x_ip^i$ et on le fait 
de proche en proche, $h^{p^{\alpha-1}}=(g^{x})^{p^{\alpha-1}}$.
Alors\\
\begin{center}
    Déterminer $x_0$ c'est déterminer le log discret 
    de $h^{p^{\alpha-1}}\mod p$.
\end{center}
\noindent Maintenant étant donnés $x_0,\ldots, x_{e-1}$ on calcule 
$x_e$, on fait comme pour shanks tonelli, i.e.

    \[h_{e-1}^{p^{\alpha-(e+1)}}=(g^{p^{\alpha-1}})^{x_e}\]

\textbf{Complexité totale (selon Shoup):} $\Or(rlog^)$


\section{Tests de primalité}
Plusieurs types: 
\begin{enumerate}
    \item test de composition
    \item test de primalité
    \item (Cribles)
\end{enumerate}

On considère tjr un entier impair(obvious). Et le test 
naif ou on teste tout les entiers plus petits: $\Or(\sqrt{n}\log^2(n))$

\subsection{Tests probabilistes}
\indent \textbf{Test de Fermat:} On test si 
$a^{n-1}\equiv1\mod n$ pour pleins d$'a$.
\begin{defn}
    On prend $b\in(\Z/n\Z)^{\times}$. Si $b^{n-1}\equiv1
    \mod n$ alors $n$ est dit pseudo-premier en base $b$.
\end{defn}

\begin{prop} eq\\
    \begin{itemize}
        \item[(i)] $n$ est p.p en base $b$ ssi $\lvert 
        b\rvert\mid n-1$
        \item[(ii)] Si $n$ est p.p en base $b_1$ et $b_2$
        alors $n$ est pp en base $b_1b_2$
        \item[(iii)]Si $\exists c\in (\Z/n\Z)^{\times}$ tq 
       $c^{n-1}\neq 1\mod n$, alors il ya au moins autant
       de bases pour lesquelles $n$ n'est pas pp que 
       de bases pour lesquelles il l'es.
    \end{itemize}
\end{prop}
\textbf{Démo: (iii)} Etant donné $B=\{b|b^{n-1}\equiv 1
\mod n\}$. Supposons qu'on a ce $c$, on pose $C=cB$. On a 
$\#C=\#B$. Mtn si $cb\in B$ alors par (ii) $c\in B$. Pas 
possible.\qed\\

\newpage
\textbf{Algo:} On itère $k$ fois

\begin{itemize}
    \item on tire aléatoirement $b$.
    \item On teste si $b$ est premier à $n$.
    \item Si oui on teste si $b^{n-1}\equiv 1\mod n$.
    \begin{itemize}
        \item[Non] $n$ est composé.
        \item[Oui] On change de $b$.
    \end{itemize}
\end{itemize}
Si après $k$ tirages, l'algo n'indique que $n$ est pas composé 
alors il est premier avec une proba $\geq 1-\frac{1}{2^k}$
à la condition qu'il existe un $c$ vérifiant (iii).\\

\textbf{Complexité}: \(\Or(k\log^3 n)\).
\begin{defn}
    Les entiers de Carmichael sont ceux qui vérifient 
    Fermat en étant composé impairs. 
\end{defn}
\noindent\begin{prop}
    \begin{itemize}(eq)
        \item[(i)]Si $n$ contient un facteur carré 
        alors $n$ n'est pas de Carmichael.
        \item[(ii)]Si $n$ est sans facteurs carré,
        $n$ est de Carmichael ssi $\forall p\mid n$
        \[(p-1)\mid (n-1)\]
    \end{itemize}
\end{prop}

\textbf{Proof: (i)}
Avec \(<g>=(\Z/p^2\Z)^{\times}\) et 
\begin{itemize}
    \item $b\equiv g \mod p^r$
    \item $b\equiv 1\mod n/p^r$
\end{itemize}

D'ou $p(p-1)\mid n-1$ et comme $n-1\equiv 1\mod p$.\qed

\begin{thm}[Alford, Pomerance et Granville]
    Il y'a une infinité de nombres de Carmichael. (par 
    exemple \(561=3*11*17\))
\end{thm}
\newpage
\textbf{Test de Solovay-Strassen}:
\begin{prop}
    Si $\forall b\in(\Z/n\Z)^{\times}$, on a 
    \[\left(\frac{b}{n}\right)=b^{\frac{n-1}{2}}\]
    Alors $n$ est premier.
\end{prop}
\begin{lem}
    Si $n$ est composé impair, alors pour au moins 
    la moitié des $b\in(\Z/n\Z)^{\times}$,
    \[\left(\frac{b}{n}\right)\neq b^{\frac{n-1}{2}}\]
\end{lem}
\noindent \textbf{Démo:} Même argument que pour Fermat. Y faut 
juste avoir un \(c\) comme dans l'énoncé. On le cherche
\begin{itemize}
    \item[Cas ou $p^2\mid n$:] On pose $c=1+\frac{n}{p}$.
On a $\left(\frac{c}{n}\right)=\left(\frac{c}{n}\right)^2\times
\left(\frac{c}{\frac{n}{p^2}}\right)=1.1=1$.
 Puis $c^j=(1+\frac{n}{p})^j=1+j.\frac{n}{p}\mod n$ d'ou 
 $c^j=1\mod n$ ssi $j\equiv 0\mod p$. Maintenant 
 $\frac{n-1}{2}\equiv 0\mod p$ ssi $n-1\equiv0\mod p$ qui 
 est pas possible.
    \item[Cas ou $n=\prod p_i$:] On prend $c_0$ un résidu
non quadratique pour $p_0$ et $c_1\equiv 1\mod n/p_0$. 
On prend $c$ tel que $c\equiv c_0\mod p_0$ et 
$c\equiv c_1\mod \frac{n}{p_0}$. Maintenant en utilisant 
$\left(\frac{c}{n}\right)=\left(\frac{c_0}{p_0}\right)
\times\left(\frac{c_1}{n/p_0}\right)$ on prouve que le 
$c$ est bien comme on veut.
\end{itemize}

\noindent\textbf{Algo de Solovay-Strassen:} On itère $k$ fois.
\begin{itemize}
    \item On tire aléatoirement $b\in[2,n-1]$
    \item On teste si $b$ et $n$ sont premiers entre eux 
    \begin{itemize}
        \item[Non] $n$ est composé
        \item[Oui] On teste si $\left(\frac{b}{n}\right)
        =b^{\frac{n-1}{2}}\mod n$
        \begin{itemize}
            \item[Non] $n$ est composé.
            \item[Oui] On change de $b$.
        \end{itemize}
    \end{itemize}
\end{itemize}
Après $k$ passages avec succès, $n$ est premier avec 
une proba $\geq 1-\frac{1}{2^k}$.\\

\textbf{Complexité}: $\Or(k\log^3n)$.
\newpage 
\textbf{Test de Miller-Rabin:} l'idée est que si 
$n\ne p^{\alpha}$ est composé alors $n$ a plus de $2$ racines 
carrées.

\begin{defn}
    On pose $s=v_2(n-1)$ et $t=n/2^s$. Soit $b\in(\Z/n\Z)^\times$.
    \\\indent Si $b^t\equiv 1\mod n$ ou $\exists 0\leq r<n-1$ tq 
    $(b^t)^{2^r}\equiv -1\mod n$ alors on dit que $n$ 
    est pp fort en base $b$.
\end{defn}
\begin{lem}[$n$ un entier impair composé]
    Pour au plus $1/4$ des élts de $(\Z/n\Z)^{\times}$,
    $n$ est pp fort.
\end{lem}
\textbf{Algo de Miller-Rabin:} On itère $k$ fois,
\begin{itemize}
    \item On tire aléatoirement $n\in[2,n-2]$.
    \item On test si $b$ est premier à $n$.
    \begin{itemize}
        \item[Non] $n$ est composé.
        \item[Oui] On regarde si $b^t\equiv\pm1\mod n$.
        \begin{itemize}
            \item[Si oui:] On change de $b$
            \item[Sinon:] On calcule les $(b^t)^{2^r}\mod n$
            \begin{itemize}
                \item[Un]$-1$: On change de $n$
                \item[Pas de]$-1$: $n$ est composé
            \end{itemize}
        \end{itemize}
    \end{itemize}
\end{itemize}
Après $k$ tirages avec succès, $n$ est premier avec une 
proba $\geq 1-\frac{1}{4^k}$.\\
\indent \textbf{Complexité:} $\Or(k\log^3n)$.

\subsection{Test de primalité assurée}
\textbf{Test $n-1$ de Pocklington-Lehman:}On considère 
$n$ impair.
\begin{prop}
    Soit $p\mid n-1$. Si on a un $a\in\Z$ tq:
    \[a^{n-1}\equiv 1\mod n\]    
    et ($a^{\frac{n-1}{p}}-1$) est premier à $n-1$.
    Alors pour tout diviseur $d$ de $n$, \[d\equiv1\mod 
    p^{\alpha}\] où $p^{\alpha}\mid n-1$.
\end{prop}
\indent \textbf{Démo:} Soit $d\mid n$ premier et $a$ comme dans 
l'énoncé. On a $a^{n-1}\equiv1\mod f$. Et $a^{\frac{n-1}{p}}
\neq 1\mod d$. On note $e_d=\lvert a\rvert$ mod $d$. Donc $e_d\mid n-1$
et $e_d\nmid (n-1)/p$. Si $p^{\alpha}\mid n-1$ alors $p^{alpha}
\mid e_d$. En plus $a^{d-1}\equiv 1\mod d$ d'ou $e_d\mid d-1$.
Puis $p^{\alpha}\mid e_d$ et $d\equiv 1\mod p^{\alpha}$.qed

\begin{cor}
    Soit $n$ un entier tq $n-1=f.u$ avec $f\geq\sqrt{n}$,
    $f$ complètement factorisé et $f$ premier à $u$. 
    Supposons que $\forall p\mid f$, $\exists a$ comme 
    dans la prop précedente. Alors $n$ est premier et 
    inversement.
\end{cor}
\textbf{Démo:} Supposons que $\forall p\mid f$, il existe 
un $a$ comme dans l'énoncé. D'après la prop, $\forall p\mid f$,
$\forall d\mid n$ on a $d\equiv1\mod p^{\alpha}$. D'ou 
par le lemme chinois $d\equiv 1\mod f$. Supposons que 
$d=1+kf\neq 1$. Alors $n$ n'a pas de diviseur $\neq1\leq 
\sqrt{n}$ donc $n$ est premier. Inversement, si 
$n$ est premier, on prend un générateur $a$ de $\F_n^*$.

\subsection{L'algorithme AKS}
\textbf{AKS pour Aggraval, Kayal et Sapena. Idée:} $n$ 
est premier ssi $(X+a)^n=X^n+a\mod n$. (On va regarder 
modulo un polynome)\\
\indent \textbf{Algo AKS:} On évince le cas ou $n=m^b$ et $b\ne1$.
\begin{itemize}
    \item On pose $r$ minimal tel que $\Or_r(n)\geq 4
    \log^2n$ avec ($\Or_r(n)$ l'ordre de $n$ dans $(\Z/r\Z)^{\times}$)
    \item Si $\exists1\leq a\leq r$ tq $1<a\wedge n<n$. Alors 
    $n$ est composé.
    \item Si $n\leq r$, $n$ est premier.
    \item Pour $1\leq a\leq [2\sqrt{\phi(r)}\log n]$,
    on teste si $(X+a)^n\neq X^n+a\mod (n, X^r-1)\implies$
    $n$ est composé. Sinon $n$ est premier.
\end{itemize}

\textbf{Complexité:} $\Or(\log^{10,5}(n))$.

\section{factorisation d'entiers}
La méthode naive est en $\Or(\sqrt{n}\log^2n)$ comme d'hab.
\subsection{Algo de Rho-Pollard}
\textbf{Idée:} On fait une balade aléatoire dans $\Z/n\Z$
(on calcule une suite ($x_i$)) et pour tout $x_j$ calcule
ou teste si pour $i<n$ $(x_j-x_i)\wedge n\ne1,n$. On obtient
un $r$ tel que $x_j\equiv x_i\mod r$. (On prend $x_j-x_i$ Et
pas $x_j$ pour avoir un paradoxe des anniversaires (bcp plus de poss
en ajoutant $x_i$)). \\
\begin{prop}[Rappel:]
    Une suite $f(x_i)=x_{i+1}$ de $S\to S$ ($S$ fini),
    contient une collision en $l$ étapes ou $l=1+[\sqrt{2\lambda r}]$.
    avec proba $\leq e^{\lambda}$.
\end{prop}

\textbf{Sans mémoire:} Etant donné $(i_0,j_0)$ deux premiers 
indices tq $(x_{j_0}-x_{i_0})\wedge n=r\ne1,n$. Soit $j\in\N$, on
pose $h=[\log j]$(ie $2^h\leq j<2^{h+1})$. On pose $i=2^n-1$
\indent On calcule $(x_k)$ tq $f(x_k)=x_{k+1}$ et on teste 
si $(x_j-x_i)\wedge n\ne1,n$ avec $i$ déf comme précedemment.
Si $x_{j_0}\equiv x_{i_0}\mod r$ alors \[
    x_{j_0+m}\equiv f^m(x_{j_0})\equiv f^m(x_{i_0})\equiv 
    x_{i_0+m}\mod r
    \]
Donc si $j-i=j_0-i_0$ on a $x_j\equiv x_i\mod r$. On pose 
$h_0=[\log j_0]$ et $i=2^{h_0+1}-1$ et $j=i+j_0-u_0$. Par 
dèf de $j$, $x_j\equiv x_i\mod r$: \[
    j=2^{n_0+1}-1 +j_0-i_0
\]
\[
    j_0-i_0\geq 1~~\text{donc}~~j\geq 2^{k_0+1}
\]
\[
    j_0-i_0\leq j_0<2^{k_0+1}
\]
Donc $j<2^{k_0+1}-1+2^{k_0+1}=2^{k_0+2}-1$ et 
$2^{k_0+1}\leq j<2^{k_0+2}$ ($[\log j]=h_0+1$)\\
\newline
\indent \textbf{La fonction $f$ est arbitraire faut bien
 la choisir}

\subsection{Méthode de Fermat et améliorations}
\begin{prop}
    Il existe une bijection entre les ocupes $(a,b)$ tq 
    $x=ab$ et les couples $(t,s)$ tq $x=t^2-s^2$.
\end{prop}

\textbf{Idée:} On pose $t=[\sqrt{n}]+1$ et on teste si 
$t^2-n$ est un carré. Sinon on passe à $t+1,t+2,\ldots$

\textbf{Méthode de Dixon}\\
\textbf{Idée:} Au lieu de travailler dans $\Z$, on travaille
dans $\Z/n\Z$. On cherche $t^2\equiv s^2\mod n$. Si on 
a un couple, on calcule $(t-s)\wedge n$. Comme c'est dur 
de trouver ces couples on les construits.\\
\textbf{Méthode:}\\
On décide d'une base de facteurs $B\subseteq\Z$  pour 
laquelle on va chercher des $t$ tq $t^2\mod n$ soit 
$B$-friables i.e. $t^2=\prod_{b_i\in B}b_i^{\alpha_i}\mod n$.
\begin{itemize}
    \item On choisit $A\in\Z$ ou on va tester si $t\in A$ est tq 
$t^2\mod n$ est $B$-friable.
    \item On teste si $\forall t\in A$, $t^2\mod n$ est 
$B$-friable. On obtient ainsi $A'\subseteq A$ où pour 
tout $t\in A'$, $t^2\mod n$ est friable.
    \item Pour tout $t_j\in A'$, on associe \[
        \epsilon_j=(\alpha_{1j}\mod 2, \alpha_{2j}\mod 2,
        \ldots,\alpha_{bj}\mod 2)
    \]
    \item A partir de ces $\epsilon_j$ on cherche $\sum_j \epsilon_j=0$
    \[
        \prod_j t_j^2=\prod\prod_{b_i\in B}b_i^{\alpha_{i,j}}
        =\prod_{b_i\in B}=b_i^{\sum \alpha_{i,j'}}\mod n
    \]
\end{itemize}

\noindent \textbf{Crible quadratique:}
\begin{itemize}
    \item $B=\{p| p<P~\text{et}
\left(\frac{n}{p}\right)=1\}\cup\{2\}$, on pose $\#B=h$.
    \item $A=\big[[\sqrt{n}]+1,[\sqrt{n}]+K\big]$ où
$P<K<P^2$.
\end{itemize}
Dans ce crible on va tester si $t^2 - n$ est $B$-friable 
avec $t\in A$. Si $t^2-n\mod[n]=p_i^{\alpha_i}
\prod_{i\ne1}p_l^{\alpha_l}$ avec $\alpha_i\geq 1$. Alors 
$t^2-n\equiv 0 \mod[p_i^{\alpha_i}]$ donc $t^2\equiv n\mod[p_i^{\alpha_i}]$.
Si $t$ est tq $t^2\equiv n\mod[p_i]$ alors $(t+kp_i)^2\equiv n \mod[p_i]$.\\

\noindent \textbf{suite:} Soit $2\ne p\in B$. Par def $\left(\frac{n}{p}\right)=1$.
On calcule $x_0$ tq $x_0^2\equiv n\mod[p]$. On cherche 
$x_1$ tq $(x_0+x_1p)^2\equiv n\mod[p^2]$ puis 
\[
    x_0^2+2x_0x_1p+x_1^2p^2\equiv n\mod[p^2]
\]
\[
    p(k_0+2x_0x_1)\equiv0\mod[p^2]
\]
donc $k_0+2x_0x_1\equiv 0\mod[p]$. On relève la racine 
tant qu'il existe $t\in A$ tq $t^2\equiv n\mod[p^{\alpha}]$

\textbf{Complexité:} \(\Or(e^{(1+\epsilon)\sqrt{\log n \log\log n}})\)

\subsection{}

\end{document}
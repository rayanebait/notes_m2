
\documentclass[12pt]{article}
\usepackage[dvipsnames]{xcolor}
\usepackage{hyperref, pagecolor, mdframed}
\usepackage{graphicx, amsmath, latexsym, amsfonts, amssymb, amsthm,
amscd, geometry, xspace, enumerate, mathtools}
\usepackage{tikz}

\theoremstyle{plain}
\newtheorem{thm}[subsubsection]{Th\'eor\`eme}
\newtheorem{lem}[subsubsection]{Lemme}
\newtheorem{prop}[subsubsection]{Proposition}
\newtheorem{propr}[subsubsection]{Propri\'et\'e}
\newtheorem{cor}[subsubsection]{Corollaire}
\newtheorem{intro}[section]{Introduction}
\newtheorem{thm2}[subsection]{Th\'eor\`eme}
\newtheorem{lem2}[subsection]{Lemme}
\newtheorem{prop2}[subsection]{Proposition}
\newtheorem{propr2}[subsection]{Propri\'et\'e}
\newtheorem{cor2}[subsection]{Corollaire}

\theoremstyle{definition}
\newtheorem{defn}[subsubsection]{D\'efinition}
\newtheorem{rmq}[subsubsection]{Remarque}
\newtheorem{conj}[subsubsection]{Conjecture}
\newtheorem{exmp}[subsubsection]{Exemples}
\newtheorem{quest}[subsubsection]{Exercices}
\newtheorem{defn2}[subsection]{D\'efinition}
\newtheorem{rmq2}[subsection]{Remarque}
\newtheorem{conj2}[subsection]{Conjecture}
\newtheorem{exmp2}[subsection]{Exemples}
\newtheorem{quest2}[subsection]{Exercices}

\theoremstyle{remark}
\newtheorem{rem}{Remarque}
\newtheorem{note}{Note}

\newcommand{\M}{\mathcal{M}}
\newcommand{\A}{\mathcal{A}}
\newcommand{\N}{\mathbb{N}}
\newcommand{\Z}{\mathbb{Z}}
\newcommand{\R}{\mathbb{R}}
\newcommand{\F}{\mathbb{F}}
\newcommand{\Or}{\mathcal{O}}
\newcommand{\gr}{\color{Sepia}}
\newcommand{\rg}{\color{Red}}

\hypersetup{
    colorlinks=true,
    linkcolor=blue,
    urlcolor=Green,
    filecolor=RoyalPurple
}

\definecolor{wgrey}{RGB}{148, 38, 55}


\title{Cryptographie}
\date{26 septembre 2023}
\begin{document}
\maketitle

\section{Codes}
\subsection{Définitions}
\begin{defn}
    La distance de Hamming sur $\F_q^n$ est donnée par $d(x,y)=\#\{i|x_i\ne y_i\}$.
\end{defn}
\begin{defn}
    Un \([n,k]_q\) code linéaire est un sev de $\F_q^n$ de dim $k$. Un $[n,k,d]_q$ code 
    définit une distance minimale des élts du sev.
\end{defn}
\begin{defn}
    On peut def un code par une matrice génératrice
    est une matrice dont les colonnes engendre le code.
    (on la suppose de taille $n*l$, $l=k$.)
\end{defn}

\begin{defn}
    On peut aussi def par une matrice de parité. (i.e. $[n,k]_q=\ker(H)$)
\end{defn}
(Le nom parité vient, du cas $\F_2$.)

\begin{defn}
    Matrice génératrice systématique: \[M=[id_k, A]\]
    Elle est unique (combinais lineaire de A $\implies$ tjr une base mais pas le mm sev)
\end{defn}

\begin{defn}[Distance minimale]
    Etant donné un code $C$, $d_{C}=\inf\{d(x,y)| x\ne y\}$. Ou 
    \[
        \inf\{d(x,0)\}
    \]
\end{defn}

\begin{lem}
    Etant donné un $[n,k,d]_q$-code linéaire $C$, pour tout deux $x,y$: 
    \[
        B(x,[(d-1)/2])\cap B(y,[(d-1)/2])=\emptyset
    \]
\end{lem}

\begin{thm}[Singleton]
    $d_C\leq n-k+1$.
\end{thm}

\begin{thm}[Pas de redondance inutile]
    Les codes MDS (qui atteignent la borne) vérifient:
    \begin{itemize}
        \item Tout ensemble de $k$ colonnes d'une matrice génératrice $G$ d'un MDS 
        est inversible.
        \item Tout ensemble de $n-k$ colonnes d'une matrice de parité de $G$ d'un MDS 
        est inversible.
    \end{itemize}
\end{thm}

\section{Codes étendus}
\begin{defn}
    $C$ un $[n,k]_q$-code tel que $\exists c\in C$, $\sum c_i\ne 0$. Le code 
    étendu de $C$ est \[Ext(C)=\{(c_1,\ldots, c_n, -\sum c_i)|(c_i)\in C\}\]
\end{defn}
\begin{prop}
    $H'=\begin{pmatrix}
        H&&0\\
         &&\vdots\\
         &&0\\
        1&\ldots&1
    \end{pmatrix}$
    est une matrice de parité de $Ext(C)$ et $Ext(C)$ est un 
    $[n+1,k]_q$
\end{prop}
\subsection{Poinconnage}
\begin{defn}
    $C$ un $[n,k,d]_q$-code et $I\subset [1,n]$. 
    \[
        P_I(C):=\{(c_i)_{i\in[1,n]-I}\}
    \] 
\end{defn}
(On enleve des lignes de la matrice)

\textbf{Notation:} Etant donné $M$ une matrice et $I$ des indices, on note $M_I$ la matrice indexée par I.

\begin{prop}
    Soit $G\in \F_q^{k*n}$ une matrice génératrice de $C$, alors 
    $G_{i\in[1,n]-I}$ est une matrice génératrice de $P_I(C)$.
\end{prop}
\begin{prop}
    $P_I(C)$ est un $[n',k',d']_q$ code avec $n'=n-\# I$, $k'\leq K$ 
    et $d-\#I\leq d'\leq d$.
\end{prop}
\textbf{Preuve}: Pour $d'$, soit $c\in C$, alors 
\[\lvert c\rvert-\#I\leq\lvert c_{i\in[1,n]-I}\rvert \]
\subsection{Raccourcissement}
\begin{defn}
    \[R_I(C):=\{(c_i)_{i\in[1,n]-I}| c\in C~et~(c_i)_{i\in I}=0\}\]
\end{defn}
(On enleve des lignes de la matrice de parité)
\begin{prop}
    Si $H$ est une matrice de parité de $C$ alors
    $H_{[1,n]-I}$ est une matrice de parité de $R_I(C)$.
\end{prop}
\textbf{Preuve}: $H':=H_{[1,n]-I}$\\
(*) Mq $R_I(C)\in Ker(H')$. Soit $c'\in R_I(C)$, $\exists c\in C$ tq 
$c_{[1,n]-I}=c'$, or $Hc^T=0=H'c'+H_Ic_I=H'c'$
(**) Mq $Ker(H')\subseteq R_I(C)$. Soit $c'\in Ker(H')$ donc 
$H'c'^T=0$. Soit $c\in \F_q^n$ tq $c_I=0$ et $c_{[1,n]-I}=c'$. On a 
alors $Hc^T=H'c'^T+H_Ic_I=H'c'^T=0$ donc $c\in C$ et donc $c'\in R_I(C)$.

\begin{prop}
    $R_I(C)$ est un $[n', k', d']_q$-code avec
    $n'=n-\#I$, $k'\geq k-\#I$, $d'\geq d$.
\end{prop}
\textbf{Preuve}: Pour $d'$, $\{c|c_{[1,n]-I}\in R_I(C)+c_I=0\}\subseteq C$ d'ou $d'\geq d$.

\begin{prop}
    $P_I(C)^{\perp}=R_I(C)$ et $R_I(C)^{\perp}=P_I(C)$.
\end{prop}
\subsection{Subfield Subcode}
Dans la suite $m\geq1$. 
\begin{defn}
    Soit $C$ un $[n,k]_{q^m}$ code linéaire. Le subfield subcode de $C$ est 
    $C|_{\F_q}=C\cap \F_q^n$.
\end{defn}
\begin{prop}
    Si $C$ est $[n,n-r,d]_{q^m}$. Alors $C|_{\F_q}$ est $[n,\geq n-mr,\geq d]_q$.
\end{prop}
\textbf{Preuve}: $C|_{\F_q}\subseteq C$ donc $d'\geq d$. Pour la dimension
on pose \[\phi:\F_{q^m}^n\to\F_{q^m}^n\]\[(x_i)\mapsto (x_i^q-x_i)\]. On a 
$Ker(\phi)=C|_{\F_q}$. Restreindre à $C$ et conclure.

\subsection{Code trace}
\begin{defn}
    Soit $a\in \F_q^n$. \[Tr_{\F_q^m/\F_q}(a):= a+a^q+\ldots+q^{m-1}\]
\end{defn}
\begin{rem}
    Trace donnée par la mul ! (regarder une base du type $(\alpha^i)_i$)
\end{rem}

\begin{prop}
    $Tr$ est à valeur dans $\F_q$
\end{prop}
(juste appliquer frob, sinon trouver la matrice et le pol char qui sont dans $\F_q$)

\begin{prop}
    La trace est $\F_q$-linéaire, surjective et non dégénérée.
\end{prop}

\begin{prop}
    Soit $C$ un $[n,k,d]_{q^m}$ code linéaire, alors $Tr(C)$ est un 
    $[n',k',d']_q$ un code linéaire avec $n'=n$ et $k'\leq mk$.
\end{prop}

\begin{thm}[Delsarte]
    $(C|_{\F_q})^{\perp}=Tr(C^{\perp})$ et $(Tr(C))^{\perp}=C|_{\F_q}$.
\end{thm}
\textbf{Preuve:} a faire.

\section{Reed-Solomon}
\begin{defn}
    Soit $x\in \F_q^n$, $(x_i)$ deux à deux distincts, avec $n\leq q$ et soit $k\leq n$.
    Le code de Reed-Solomon associé à $x$ est 
    \[RS_k(x):=\{c=(f(x_1),\ldots,f(x_n))|f\in \F_q[X]_{<k}\}\]
\end{defn}
\begin{prop}
    Une matrice génératrice de $RS_k(x)$ est:
    $\begin{pmatrix}
        1 & 1&\ldots&1\\
        x_1&x_2&\ldots&x_n\\
        \vdots&\vdots&\ldots&\vdots\\
        x_1^{k-1} & x_2^{k-1}\ldots&x_n^{k-1}\\
    \end{pmatrix}$
\end{prop}

\begin{prop}
    Les $RS$ sont MDS.
\end{prop}
\textbf{Preuve:} On regarde $\phi_{k,x}:\F_q[X]_{<k}\to \F_q^n$ qui 
a $f$ associe $(f(x_i))_i$. Elle est injective, clair. 
Soit maintenant, $c=(f(x_1),\ldots,f(x_n))\ne 0$. Alors $f\ne 0$ et 
$f$ a au plus $k-1$ racines distinctes donc $|c|\geq n-k+1$ donc 
$d_C\geq n-k+1$ or avec singleton, on a aussi $d_C\leq n-k+1$.
\begin{defn}[GRS]
    On regarde $x=(x_1,\ldots, x_n)\in \F_q^n$ avec $n\leq q$
    et $(x_i)$ deux à deux distincts. Soit $y=(y_1,\ldots, y_n)\in\F_q^{*n}$.
    On déf \[GRS_k(x,y)=\{c=(y_if(x_i))_{i\in [1,n]}|f\in \F_q[X]_{<k}\}\]
\end{defn}
\begin{prop}
    A nouveau, les $GRS$ sont MDS.
\end{prop}
\textbf{Preuve:} Tous isomorphes, via une isométrie, à des $RS$. ($y_i$ sont non nuls)

\begin{thm}[$q<=n$]
    L'orthogonal de $RS_q(x)$ est $RS_{q-k}(x)$.
\end{thm}
\textbf{preuve a faire}: (produit scalaire)

\begin{thm}
    Maintenant si $x\in \F_q^n$ tq les $x_i$ sont deux 
    à deux distincts et $y\in (\F_q^*)^{n}$. Alors,
    \[GRS_k(x,y)^{perp}=GRS_{n-k}(x,y')\]
    avec $y_i'=\frac{-1}{y_i\prod_{i\ne j}(x_i-x_j)}$
\end{thm}
\textbf{Preuve:} Noter $Q(\alpha)=\prod_{\alpha\ne x_i}(x-\alpha)$.
Alors $y_i'=Q(x_i)/y_i$. Puis 
\[
    <(y_if(x_i))_i, y_i'g(x_i)>=\sum_{\alpha\in \F_q} Q(\alpha)f(\alpha)g(\alpha)
\]
Et on se ramène au $RS$ normal.

\subsection{Décodeurs uniques}
-Berlekamp-Welch: interpolation.\\
-Euclide etendu: Berlekamp-Massey (vision BCH).\\

On corrige au plus $[(n-k)/2]$ erreur.

\subsection{Décodage en liste}
A part quelques exception, les codes sont généralement
pas parfait (on atteint pas la borne de Hamming). On a 
meme souvent que l'union des boules centrées sur le code
de rayon $t=(d_C-1)/2$ ne représentent qu'une petite partie 
de l'espace ambiant. On peut généralement décoder au 
delà de $(d_C-1)/2$.\\

\textbf{Idée:} On va décoder au delà de $[(d_C-1)/2]$.
Généralement on a seul mot dans la liste de décodage.
Si on en a plusieurs, on teste lequel est le plus proche 
du mot reçu.Pour que le décodeur reste en temps polynomial
, il faut que la liste soit de taille polynomiale.

\begin{thm}[Borne de Johnson]
    Soit $C$ un $[n,Rn, \delta n]_q$ un code linéaire. 
    Pour $R,\delta$ des constantes. Soit 
    \[\rho =(1-\frac{1}{q})\left(1-\sqrt{1-\frac{q\delta}{q-1}}\right)\]
    Alors $\forall a\in \F_q^n$,
    \[\#(B(y,\rho n)\cap C)\leq q\delta n^2\]
\end{thm}

\begin{rem}
    Pour $q=2$, \[\rho=\frac{1}{2}(1-\sqrt{1-2\delta})\]\\
    Pour $q\to \infty$, on a 
    \[\rho=(1-\sqrt{1-\delta})\]
    En particulier, pour un GRS, lorsque $n\to\infty$
    et donc $q\to \infty$, on a 
    \[\rho=1-\sqrt{R}\] 
\end{rem}


\section{Codes LDPC}

\begin{lem}[Pilling-Up-Lemma]
    $(p_i)$, ($p_i=P(b_i=1)$, $b_i$ 
    une variable aléatoire dans $\F_2$). Soit $y=c+e$ avec $c\in C$ et $e_i\leftarrow Bernoulli$:\\
    \indent Pour tout $h\in C^{\perp}~t.q.~|h|=w$
    \[P(<y,h>=0)=(1+\prod_{i\in Supp(h)}(1-2Pi))\]
\end{lem}
(Dans le pdf)
En gros l'application c'est que on suppose les 
équations de parités n'ayant pas de $1$ en commun. 
(Conditionnellement indépendants par rapport à b)

\textbf{Questions:}
\begin{enumerate}
    \item Etant donné un code LDPC et une matrice de parité 
quelconque, est-ce qu'on peut trouver une matrice de 
parité creuse ?
    \item $E(\#(mots~de~poids~w~dans~code~dual~aleatoires)):=
    E(w)=2^{n-k}binom(n,w)/2^n$
    
\end{enumerate}

\end{document}

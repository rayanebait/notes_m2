
\documentclass[12pt]{article}
\usepackage[dvipsnames]{xcolor}
\usepackage{hyperref, pagecolor, mdframed }
\usepackage{graphicx, amsmath, latexsym, amsfonts, amssymb, amsthm,
amscd, geometry, xspace, enumerate, mathtools}
\usepackage{tikz}

\theoremstyle{plain}
\newtheorem{thm}[subsubsection]{Th\'eor\`eme}
\newtheorem{lem}[subsubsection]{Lemme}

\theoremstyle{definition}
\newtheorem{defn}[subsubsection]{D\'efinition}

\theoremstyle{remark}

\newtheorem{rem}{Remarque}

\hypersetup{
    colorlinks=true,
    linkcolor=blue,
    urlcolor=Green,
    filecolor=RoyalPurple
}

\definecolor{wgrey}{RGB}{148, 38, 55}


\title{Protocoles réseaux : cours1}
\date{02 octobre 2023}
\begin{document}
\maketitle

\section{Rappels}
Cours m1 :
$$\text{application}\ne\text{protocole}$$

Protocoles de couche application :
\begin{itemize}
    \item PC$\rightarrow$serveur de mail, deux protocoles : SMTP, IMAP4.
    \item tel$\rightarrow$serveur de mail : SMTP, IMAP4.
    \item Fair Email$\rightarrow$serveur de mail : SMTP, IMAP4.
\end{itemize}

$\ne$
\begin{itemize}
    \item whatsapp ou on peut parler qu'en whatsapp (le client est sur l'appli)(probleme : c'est un protocole ET une appli)
    \item zulip : on peut faire son propre protocole pour echanger avec le serveur mais le protocole zulip change a chaque vers
\end{itemize}

\subsection{Couches}
Les différentes couches :
\begin{enumerate}
    \item Physique
    \item Lien
    \item Reseau
    \item Transport
    \item (7) Application
\end{enumerate}

En bas : le materiel. En haut : utilisateur.\\
\section{Internet est cassé}
\begin{center}
    \textbf{ "internet est cassé"}.
\end{center}
\begin{enumerate}
    \item Il est "Ossifié"
    \item "Tout le monde vous en veut"
\end{enumerate}

\subsection{"Ossification"}
\begin{rem}
    Le routeur a que trois couches, il depasse pas IP et regarde pas les paquets. Permet de changer les protocoles du dessus.
\end{rem}

Le problème c'est que certains le font.
\begin{defn}
    Middlebox : entité au seins du réseau (pas aux bouts, typiquement au dessus de la troisieme couche d'un routeur) qui regarde à l'interieur des paquets.
\end{defn}

\textbf{Ex :} \begin{itemize}
    \item Firewalls : voit ce qui respecte pas les protocoles ? Et plus mais sert a rien askip. 
    \item En pratique, le firewall jette tout ce qui est pas TCP, UDP. Problématique.
    \item NAT, pareil jette les paquets d'autres protocoles.
    \item accélerateurs
    \item black boxes (déployé par les services internet du a ansi/DGSE/autres, espionnage)
    \item NAT$\rightarrow$limitations de la connectivité. 
\end{itemize}

\begin{lem}
    Pour programmer sur le vrai internet :
    \begin{itemize}
        \item Faut utiliser TCP/UDP.
    \end{itemize}
    Faut traverser le NAT/Firewall.
\end{lem}

Ducoup on utilise des protocoles qui traversent le NAT et le Firewall.
\begin{lem}
    Http/https traversent le NAT/Firewall
\end{lem}

\begin{thm}
    La vraie couche de convergence c'est pas IP c'est http/https.
\end{thm}

\begin{rem}
    C'est un protocole nul : requete/reponse avec 500 octets d'entete a chaque paquet. Autre triche :\begin{itemize}
        \item Http utilise un autre protocole de transport encapsulé en UDP, quick ou microTP !
        \item ducoup on considère UDP en couche 3.5 et ducoup 4 octets par paquet au lieu des 500 de http.
    \end{itemize}
\end{rem}

Questions de \textbf{Sécurité}.

\section{Web et http}
(Web = application, http = protocole application)

\begin{defn}
    Je cite "Le Web c'est un hypertexte distribué":
    \begin{enumerate}
        \item hypertexte : contient des hyperlien (lien qui envoient ailleurs dans le fichier)
        \item distribué : les hyperliens envoient dans ailleurs ("impossible" de gerer les liens qui vont nul part $\rightarrow$ error 404 (on s'en fout))
    \end{enumerate}
\end{defn}

\subsection{http/0.9}
\begin{itemize}
    \item Ressources identifiées par des URL : "{\color{blue}http}:{\color{Green}//www.irif.fr/}{\color{red}~jch/enseignement}
    \item En rouge le chemin : http $\rightarrow$ Client envoie GET(chemin) au serveur qui renvoie la ressource a l'adresse.
    \item Plein de defauts : le serveur fait RIGHT jusqu'a la fin FIN et close(fichier). Probleme si le serveur plante et coupe la co 
    on peut pas savoir si c'est une interruption ou un fichier court.
\end{itemize}

\begin{rem}
    \begin{enumerate}
        \item TCP est lent au demarrage (petite fenetre qui grandit au fur et a mesure et devient efficace)
        \item A chaque GET, http ouvre une connexion TCP (PB serait : TCP=couche 4, http=couche 7, il faudrait une autre couche)
        \item pb quand il y a eu des img dans des fichiers (chaque balise IMG engrengeait un GET)
    \end{enumerate}
\end{rem}


{\color{blue}http}:{\color{Green}//www.irif.fr/}{\color{red}~jch/enseignement}


\subsection{http/1.0}
On a maintenant plus de requetes
\begin{enumerate}
    \item GET
    \item POST
    \item PUT
    \item -version
\end{enumerate}

Par exemple : GET /index.html .HTTP/1.0 permet d'obtenir /index.html avec une version 1.0 ou anterieure.\\
Nouvelles entetes :
\begin{itemize}
    \item Content-Type : text/html (possibilité de faire attention a pas lire un fichier pas .html en html et avoir d'autres extensions en particuliers images)
    \item Content-Length : résoud un des pbs
\end{itemize}

\subsection{http/1.1}
pas mal
\begin{itemize}
    \item entete en type texte
    \item terminaison fiable(chunked)
    \item transferts multiples sur 1 connexion
\end{itemize}

A ce moment la $\rightarrow$ : -Javascript (scripts dans les pages web)(pas forcément utile en soit).\\
\begin{itemize}
    \item XMLHttpRequest permet de faire des requete Http depuis le script !
\end{itemize}

Ducoup application web !
\begin{thm}
    Javascript + XMLHttpRequest $\rightarrow$ Web 2.0
\end{thm}

AJAX : L'interface est plus gerée par l'utilisateur.
\begin{enumerate}
    \item Le serveur envoie une page html
    \item le programme js envoie des requetes de \textbf{structures de données} (point important)
    \item le serveur ne renvoie pas du html mais des données qui permettent de remplir la page html
\end{enumerate}

XMLHttpRequest devient fetch.

\subsection{http/2}
mauvais
\begin{itemize}
    \item entete en binaire (trop de requetes) (moins lourd : algo de compression)
    \item multiplexage : requetes/reponses simultanées $\rightarrow$ pas compatible avec TCP
\end{itemize}

\subsection{http/3}
Tres bien mais compliqué : que Google qui utilise car les seuls qui ont reimplementé
\begin{itemize}
    \item UDP (quick)
    \item multiplexage efficace
\end{itemize}

\subsection{Sérialisation/désérialisation}
http transmet des octets : java/js/python utilisent des structures de données. 
\begin{itemize}
    \item Il faut traduire en octets les structures de données
\end{itemize}

Comment sérialiser/déserialiser
\begin{enumerate}
    \item Ad hoc : LL(1) à la main (faire une grammaire)
    \item XML : nul 
    \item JSON : \begin{itemize}
        \item [1,2,3] (sous ensemble de la syntaxe js)
        \item string utf-16 (pas de binaire)
        \item nombre (javascript, pas tout)
        \item tableaux
        \item dictionnaires (clés sont forcément des chaines)
    \end{itemize} Bien mieux que XML mais tjr pas super. Le binaire doit être géré ad hoc, y faut aussi sérialiser les structures
    \item Formats binaires \begin{enumerate}
        \item Ad hoc
        \item génériques : CBOR
        \item Protobuf : Structures $\rightarrow$ codage ad hoc a la compilation (dependance) $\rightarrow$ tres instable (si on change la struct le code peut etre completement different)
    \end{enumerate}
\end{enumerate}



but : protocoles qui marche aussi bien pour une appli web que native
\section{Structure des API}
Pour faire une interface Web : faut ecrire du html. Pour faire une base de donnée SQL : faut ecrire du SQL en html
\begin{enumerate}
    \item POST /sql.php HTTP/1.1
    \item HOST sql.example.com
    \item Content-type : application/sql
    \item Content-Length: 18
    \item DROP TABLE USERS (TABLE USERS est la table sql, TABLE USERS est stockée dans la RAM)
\end{enumerate}
Interet de ça : on utilise toute la puissance de la base de donnée. pb : fuite de l'implémentation. Mieux :
\begin{enumerate}
    \item POST /sql.php?query=DROP\%20TABLE...
    \item DELETE /table/users
\end{enumerate}

\subsection{Des APIS}
SOAP :
\begin{itemize}
    \item  complexe
    \item utilisant XML ($\rightarrow$ XMC-RPC)
\end{itemize}
requetes enormes codées en XML "horrible"\\
REST :
\begin{itemize}
    \item API représente un ensemble d'états (Objets sans methodes) côté serveur
    \item Si on a un serveur d'impression pour imprimer on fait pas une requete imprimer un fichier $\rightarrow$ on créer un objet impression
    \item Pas de codage (on stocke les données telle quelles, pas d'agrégation, un GET pour chacun des attributs)
    \item etats identifiés par des URL (diff entre DROP TABLE USERS $\leftrightarrow$ DELETE /table/users)
    \item opératons codées dans la méthode 
    \item Pas d'état de session (pas besoin de se souvenir de chaque client)
\end{itemize}

lulu.informatique.univ-paris-diderot.fr :8443

La lumière est "glacialement lente" : Cacher la latence de l'univers,
\begin{itemize}
    \item batching(lots de requetes) $\rightarrow$ Pas de REST. (probleme de gestion des erreurs, si la premiere requete est refusée on peut plus arreter)
    \item pipelining.
    \item prevoir l'avenir .(sur un mmo les npc apparaissent d'abord a un endroit predit par l'ordi puis ailleurs une fois que le serveur repond.)
    \item Cache ! (copie des données, attention a savoir quelles données sont celles à jour)(pb d'invalidation du cache)
\end{itemize}
\subsection{REST-fut ou REST-like}

\subsection{Tout le contraire $\rightarrow$ WebSocket}

\section{Proxys et caches}
\subsection{Localisation des caches}

\begin{itemize}
    \item Cache application. (le chat du site qui se met a jour que au reload (les msgs sont en caches))
    \item Cache navigateur. (partagé entre les pages)
    \item Proxy
\end{itemize}

Digression : proxy(censé être proche du client)(serveur mandataire)\\

Apparemment pas d'interco en couche 4 à 7.
Interconnexion \begin{enumerate}
    \item couche 1 : prise/hub 
    \item couche 2 : switch
    \item couche 3 : routeur
    \item couche 4, 7 (plus "bout en bout") : proxy
\end{enumerate}

Types de proxy HTTP :
\begin{itemize}
    \item proxy traditionnel : explicitement configuré
    \item proxy transparent : le client à aucune idée du proxy, par exemple le routeur envoie à un proxy. Pb, c'est un middlebox
    $\rightarrow$ mal. Par interception
    \item proxy inverse permet : \begin{enumerate}
        \item plusieurs serveurs, répliqués, distincts.
        \item politiques de securité
        \item cache (crétin?)
    \end{enumerate}
\end{itemize}


\subsection{proxy inverse : CDN}
Au moment d'une nouvelle maj ios : 500 000 000 de maj de 10 GO. Stratégies ? Mettre des proxys qui ont la maj PARTOUT.
Pb en dehors des majs les proxys sont loués? Solution :
\begin{itemize}
    \item  Il y a des boites de location de reseau de proxys avec cache : akamai (host www.apple.com)
\end{itemize} 
70\% du reseau passe par akamai !

\subsection{Validations}
Une requete HTTP qui semble aller direct au serveur passe en fait par au plus 5 caches (proxy, CDN, proxy inverse).
En HTTP il y'a 3 validateurs (metadonnée associée au paquets : si il a changé la donnée est pas reutilisable) :
\begin{itemize}
    \item URL (si sur un blog on change une entrée, comment on met a jour tout les utilisateurs ? On peut modifier l'URL !)
    \item Last-Modified (Pas précis, a pas utiliser)
    \item Etag (meme etag alors on peut reutiliser)
\end{itemize}
Pas de Etag, pas de Last-Modified su rle document : on utilise pas de cache.
\subsection{Controle de Cache}
(Ctrl+R, pas de revalidation. Ctrl+shift+R, revalidation de bout en bout)\\
On veut pas forcément la dernière version systématiquement. Par exemple un logo sur un site.
\begin{itemize}
    \item l'entete Cache-Control : \begin{enumerate}
        \item no-cache (ctrl+r)
        \item max-age en seconde, update le cache si la donnée est plus veille que max-age.
        \item Pas d'entete : \begin{itemize}
            \item si POST$\rightarrow$ On peut pas mettre en cache
            \item si GET$\rightarrow$ On peut mettre en cache
        \end{itemize}
    \end{enumerate}
\end{itemize}

Pour la revalidation:
\begin{enumerate}
    \item HEAD : serveur renvoie le Etag
    \item if meme Etag : GET
\end{enumerate}
Problème, faut attendre 2 requetes. Sol : GET conditionnel
\begin{enumerate}
    \item GET (header : if-None-Match) : retourne les données ou retourne none-changed
\end{enumerate}



\section{Notifications asynchrone}
Pas de notifications asynchrones : Le serveur n'envoie rien sans requetes.

\section{web socket}
API REST, etc en : http et udp. Ou tcp : flot d'octets 1 par 1. Gros problème.
\begin{itemize}
    \item Web Socket : On utilise le protocole qu'on veut ! On peut faire des flots d'objets JSON.
    \item Probleme des Web socket :\begin{enumerate}
        \item perte du cache
        \item perte de la resistance
        \item perte du load balancing
    \end{enumerate}
\end{itemize}






\end{document}
\documentclass[12pt]{article}
\usepackage[dvipsnames]{xcolor}
\usepackage{hyperref, pagecolor, mdframed }
\usepackage{graphicx, amsmath, latexsym, amsfonts, amssymb, amsthm,
amscd, geometry, xspace, enumerate, mathtools}
\usepackage{tikz}

\theoremstyle{plain}
\newtheorem{thm}[subsubsection]{Th\'eor\`eme}
\newtheorem{lem}[subsubsection]{Lemme}

\theoremstyle{definition}
\newtheorem{defn}[subsubsection]{D\'efinition}

\theoremstyle{remark}

\newtheorem{rem}{Remarque}

\hypersetup{
    colorlinks=true,
    linkcolor=blue,
    urlcolor=Green,
    filecolor=RoyalPurple
}

\definecolor{wgrey}{RGB}{148, 38, 55}


\title{Protocoles réseaux : sécurité}
\date{30 octobre 2023}
\begin{document}
\maketitle
\begin{enumerate}
    \item Politiques de sécurité (ce que/qui je veux empecher de faire quoi)
    \item Modèle d'attaque (ce que l'attaquant a le droit de faire)
\end{enumerate}

\noindent \textbf{Propriétés de sécurité} qui définissent les politiques de securité :
\begin{enumerate}
    \item confidentialité \begin{enumerate}
        \item anonymat/"méta-données"
        \item 
    \end{enumerate}
    \item authenticité
    \item integrité
    \item disponibilité, absence de déni. (par ex : un serveur doit être accessible)
\end{enumerate}

\noindent \textbf{Types d'authentifications}:
\begin{enumerate}
    \item chiffrage ad hoc, clé négociée, pas d'auth (pb de man in the middle)
    \item TOFU "trust on first use", leap of faith
    \item authentification certifiée (mac)
\end{enumerate}

\section{SSL/TLS et https}
HTTP: pas de mécanisme de sécurité $\rightarrow$ \begin{itemize}
    \item SSL$\rightarrow$ auth, chiffrage
\end{itemize}

HTTPS: HTTP+auth(du serveur)+confidentialité.




\end{document}
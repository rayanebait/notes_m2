\documentclass[12pt]{article}
\usepackage[dvipsnames]{xcolor}
\usepackage{hyperref, pagecolor, mdframed }
\usepackage{graphicx, amsmath, latexsym, amsfonts, amssymb, amsthm,
amscd, geometry, xspace, enumerate, mathtools}
\usepackage{tikz}

\theoremstyle{plain}
\newtheorem{thm}[subsubsection]{Th\'eor\`eme}
\newtheorem{lem}[subsubsection]{Lemme}
\newtheorem{prop}[subsubsection]{Proposition}
\newtheorem{propr}[subsubsection]{Propri\'et\'e}
\newtheorem{cor}[subsubsection]{Corollaire}
\newtheorem{intro}[section]{Introduction}
\newtheorem{thm2}[subsection]{Th\'eor\`eme}
\newtheorem{lem2}[subsection]{Lemme}
\newtheorem{prop2}[subsection]{Proposition}
\newtheorem{propr2}[subsection]{Propri\'et\'e}
\newtheorem{cor2}[subsection]{Corollaire}

\theoremstyle{definition}
\newtheorem{defn}[subsubsection]{D\'efinition}
\newtheorem{rmq}[subsubsection]{Remarque}
\newtheorem{conj}[subsubsection]{Conjecture}
\newtheorem{exmp}[subsubsection]{Exemples}
\newtheorem{quest}[subsubsection]{Exercices}
\newtheorem{defn2}[subsection]{D\'efinition}
\newtheorem{rmq2}[subsection]{Remarque}
\newtheorem{conj2}[subsection]{Conjecture}
\newtheorem{exmp2}[subsection]{Exemples}
\newtheorem{quest2}[subsection]{Exercices}

\theoremstyle{remark}
\newtheorem{rem}{Remarque}
\newtheorem{note}{Note}

\newcommand{\M}{\mathcal{M}}
\newcommand{\A}{\mathcal{A}}
\newcommand{\N}{\mathbb{N}}
\newcommand{\Z}{\mathbb{Z}}
\newcommand{\R}{\mathbb{R}}
\newcommand{\Or}{\mathcal{O}}
\newcommand{\gr}{\color{Sepia}}
\newcommand{\rg}{\color{Red}}

\hypersetup{
    colorlinks=true,
    linkcolor=blue,
    urlcolor=Green,
    filecolor=RoyalPurple
}

\definecolor{wgrey}{RGB}{148, 38, 55}


\title{Cryptographie}
\date{26 septembre 2023}
\begin{document}
\maketitle

On suit le introduction to modern cryptography de katz et lindell je crois. (celui que j'ai en partie lu)

\section{formalisme}
$\M$ un ensemble.

\subsection{Schéma de signature numérique}
\begin{defn}
    Triplet d'algorithme probabilistes, $\Pi$,(classe avec trois élts) "efficaces"($1/10$ de sec).
\end{defn}

\begin{itemize}
    \item KeyGen(void) : retourne ($s_k,v_k$), clé de signature(privée), clé de vérification(publique parfois).
    \item sgn Sign($s_k$, m) prend clef privée et msg et renvoie une signature $\sigma$.
    \item bool Verify($v_k$, m, $\sigma$) renvoie vrai ou faux.
\end{itemize}

Tout ça tel que $\forall~m\in\M$, si $(s_k,v_k)\leftarrow KeyGen()$
$$Verify(v_k, m, Sign(s_k,m))=1$$


\subsection{Securité $(\epsilon, t)$}
A éviter :
\begin{itemize}
    \item Manque de formalisme.
    \item Avoir un modèle trop fort qui ne peut être vérifié (il y a des attaques structurelles)(Une personne a 50 km d'une centrale est une attaque)
    \item modèle trop faible.(Un groupe de personnes armée attaque la centrale pas considéré comme une attaque)(ne prend pas en compte des attaques réalistes)
\end{itemize}

\begin{defn}
    Def bancale :$1.$ $\Pi$ est sécurisé si tout individu malveillant qui est pas censé pouvoir signer ne peut pas signer.
\end{defn}

A defs "qui est pas censé pouvoir signer".

\begin{defn}
    $2.$ $\Pi$ est securisé si $\forall m\in\M$ $\forall(v_k,s_k)\leftarrow Keygen()$, il n'existe pas d'individu qui peut produire une signature $\sigma$,
    tel que $Verify(v_k,m,\sigma)=1$.
\end{defn}

\begin{defn}
    $3.$ $\Pi$ est securisé, si $\forall m\in\M$ $\forall(s_k,v_k)\leftarrow Keygen()$...
\end{defn}

bon la vrai def :($\epsilon,t)$ sécurité
\begin{defn}
    $\Pi$ est securisée si $\forall m\in\M$ . Il n'existe pas d'algo $A$ finissant en temps $t$
    tel que $$Pr(Keygen()\rightarrow(s_k, v_k) \&\& (Verify(v_k, m, A(v_k)) == 1))\leq \epsilon$$
\end{defn}

On définit $ExpInforg(A,\Pi, m)$: \\
\indent \indent-$(s_k,v_k)\leftarrow \Pi.Keygen()$\\
\indent \indent-$\sigma \leftarrow A(v_k)$\\
\indent \indent-Retourne $\Pi.Verify(v_k, m, \sigma)$\\

On réecrit la def avec:
\begin{defn}
    $$Pr(ExpInforg_{A,\Pi,m}() == 1)\leq 10^{-10}$$
\end{defn}

Un souci avec $ExpInforg$ (modèle trop faible). Si $\Pi$ était sécurisé on pourrait construire
$\Pi'$ pareil que $\Pi$ ou verify renvoie $1$ aussi si $m = \text{message compromettant}$ signé par la mauvaise personne.(il est aussi sécurisé et la un message absurde est vérifié)
(pas sur d'avoir compris) (en gros $m$ est choisi en amont)


Je crois qu'on dit que on choisit plus $m$ a l'avance c'est $A$ qui le génère.\\
\newline

On redéfinit $ExpInforg_{A,\Pi}()$: \\
\indent \indent-$(s_k,v_k)\leftarrow \Pi.Keygen()$\\
\indent \indent-$(m, \sigma)\leftarrow A(v_k)$\\
\indent \indent-Retourne $\Pi.Verify(v_k, m, \sigma)$\\
\newline

On sait aussi que dans le monde réel l'attaquant peut avoir accès à des paires $(m,\sigma)$ particulières.(Il suffit que quelqu'un signe un msg alors une paire est disponible, trivialement)\\

\indent Le pb vient quand on arrive à signer un msg qui n'a pas a être signé. \\
\newline

On redéfinit $ExpInforgdyn_{A,\Pi}()$: \\
\indent \indent -$(s_k,v_k)\leftarrow \Pi.Keygen()$\\
\indent \indent -$Q=\emptyset$ qui stockera tout les messages signés.(variable globale)\\
\noindent On ajoute un algorithme (l'oracle)
\begin{itemize}
    \item $O_{s_k}(m)$ :
    \item $Q = \{m\}\cup Q$ 
    \item $\sigma\leftarrow \Pi.Sign(s_k, m)$
    \item retourne $\sigma$
\end{itemize}

\indent \indent-$(m, \sigma)\leftarrow A^{O_{s_k}()}(v_k)$\\
\indent \indent-Retourne $\Pi.Verify(v_k, m, \sigma) \&\& m\notin Q$\\

Ducoup la 
\begin{defn}
    $\Pi$ est $(\epsilon, t)$-securisé si il n'existe pas de $A$ tournant en temps $t$ tq 
    $$Pr((s_k,v_k)\leftarrow Keygen() \&\& ExpInforgdyn_{A,\Pi}()==1)\leq \epsilon$$
\end{defn}


\subsection{Sécurité asymptotique}
\begin{defn}
    $f:\N\rightarrow\R$ est négligeable si pour tout $k\in\N$:$$f(\lambda)=O(1/\lambda^k)$$On note $Negl$ l'ensemble des fcts negl.
\end{defn}

\begin{rem}
    On peut aussi prendre des grands $O$ et des polynomes quelconques (réels).
\end{rem}

\begin{prop}
    $Negl$ est un $\R[X]$-module.(clair)
\end{prop}

mtn la bonne def:
\begin{defn}
    Soit $\Pi_S$ un schéma de signature numérique. On dit que le schéma est sécurisé asymptotiquement si pour tout algorithme
    probabiliste polynomial $\mathcal{A}$ : $$Pr(Exp_{\Pi,\mathcal{A}}(\lambda))=neg(\lambda)$$
\end{defn}


\section{Chiffrement à clefs secrètes}

\begin{defn}
    Schéma de chiffrement à clef secrète (symétrique) : Triplet d'algo PPT
    \begin{enumerate}
        \item KeyGen($1^{\lambda}$), le paramètre de complexité est $\lambda$
        \item Enc(k, m)
        \item Dec(k, c)
    \end{enumerate}
\end{defn}

\subsection{Sécurité sémantique}
\begin{defn}
    Soit $\l\in\N$. On pose $\M=\{0,1\}^l$. Si pour tout PPT $\A$ 
    $$Pr(\A_{m\leftarrow\M}(1^l,Enc(k,m))=m^{(i)})\leq \frac{1}{2}+negl(l)$$
\end{defn}

\subsection{Sécurité eav}
On regarde le jeu \textbf{Priv$_{\A,\Pi}^{eav}(\lambda)$} :
\begin{enumerate}
    \item $k\leftarrow \Pi.Keygen(1^lambda)$
    \item ($m_0,m_1,\eta)\leftarrow\A_0$, le $\eta$ définit l'état/la mémoire de $\A$. (pas obligé de la mettre)
    \item $b\leftarrow\{0,1\}$ (choix aléatoire uniforme)
    \item $b'\leftarrow \A_1(Enc_k(m_b),\eta)$
    \item renvoie $b==b'$.
\end{enumerate}

On peut juste écrire
\begin{enumerate}
    \item $k\leftarrow \Pi.Keygen(1^lambda)$
    \item ($m_0,m_1)\leftarrow\A$ (On suppose que c'est à $\A$ de choisir les messages qu'il veut distinguer)
    \item $b\leftarrow\{0,1\}$ (choix aléatoire uniforme)
    \item $b'\leftarrow \A(Enc_k(m_b))$
    \item renvoie $b==b'$.
\end{enumerate}
On regarde aussi le jeu \textbf{Priv$_{\A,\Pi}^{eav2}(\lambda)$} :
\begin{enumerate}
    \item $k\leftarrow \Pi.Keygen(1^lambda)$
    \item ($m_0,m_1)\leftarrow\A$ 
    \item $b^*\leftarrow \A(Enc_k(m_b))$
    \item renvoie $b^*$.
\end{enumerate}

\begin{defn}
    $\Pi$ est eav-securisé si pour tout PPT $\A$ : 
    $$\lvert Pr(PrivK_{\A,\Pi}^{eav}(\lambda))-1/2\rvert=negl(\lambda)$$
    ou
    $$\lvert Pr(PrivK_{\A,\Pi}^{eav2}(\lambda,0))-Pr(PrivK_{\A,\Pi}^{eav2}(\lambda,1))\rvert=negl(\lambda)$$
\end{defn}

\subsection{exercice}
Créer un modèle de sécurité pour un gestionnaire de mot de passe.
\begin{enumerate}
    \item Init()
    \item Ajoute(B, id, mdp)->(B', b')
    \item Verify(B, id, mdp)
\end{enumerate}

experience $Exp_{\Pi,\A}()$ :
\begin{enumerate}
    \item $\beta\leftarrow$Init()
    \item $Q=\emptyset$
    \item (id, mdp)$\leftarrow\A^{\Or Ajoute_{inn},\Or Acces}$(inn pour innocent)
    \item renvoie $\Pi.Verify(\beta,id, mdp)$$\bigwedge$id$\notin Q$
\end{enumerate}
$\Or Acces()$ renvoie $\beta$ et $\Or Ajoute_{inn}(id)$ :
\begin{enumerate}
    \item mdp$\leftarrow\{0,1\}^*$
    \item ($\beta',b')\leftarrow$ $Ajoute(\beta,id,mdp)$
    \item $\beta=\beta'$
    \item renvoie $b'$
\end{enumerate}

la sécurité ducoup c'est $Pr(Exp_{\Pi,\A}(1^n)=1)<negl_n$

\section{Générateurs pseudos aléatoires}
\begin{defn}
    $l:\N\rightarrow\N$ (polynome tq $l(n)>n$).\\ On dit que $G~:~\{0,1\}^*\rightarrow\{0,1\}^*$
    un algo DPT est un géné pseudo aléatoires si ($G(\{0,1\}^n)\subset \{0,1\}^{l(n)}$):
    \begin{enumerate}
        \item $\forall$ PPT $\A$, $r\in\{0,1\}^n$ uniforme, $r'\in\{0,1\}^{l(n)}$ uniforme :
        $$\lvert Pr(\A(G(r)))- Pr(\A(r'))\rvert\leq negl_n$$
    \end{enumerate}
\end{defn}

Exos : etant donné $G(s)$ on déf 
\begin{enumerate}
    \item $G'(s)=G(\overline{s})$
    \item $G'(s)=\overline{G(s)}$
    \item $G'(s)=G(0^{\lvert s\rvert}||s)$
    \item $G'(s)=G(s)||G(s+1)$
\end{enumerate}

Maintenant les fonctions pseudo-aléatoires :
\begin{defn}
    Soit $F~:~\{0,1\}^*\times\{0,1\}^*\rightarrow\{0,1\}^*$ un DPT (meme longueur d'entrées sorties). $F$ est
    pseudo aléatoire si $\forall~D$ : $$\lvert Pr\left(D^{F_k(.)}()=0\right)-Pr\left(D^{f()}()=0\right)\rvert=negl_n$$
\end{defn}
(FPA pour fonction pseudo aléatoire)

On def $PrivK_{\Pi_{FPA},\A}^{CPA-alea}(\lambda, b)$ et $PrivK_{\Pi,\A}^{CPA}(\lambda, b)$ Ou le deuxieme c'est :
\begin{enumerate}
    \item $k\leftarrow\{0,1\}^{\lambda}$
    \item $(m_0, m_1)\leftarrow \A^{F_k()}$
    \item $r^*\leftarrow \{0,1\}^n$
    \item $b^*\leftarrow\A(r^*,m_b\oplus F_k(r^*))$, $(r^*,m_b\oplus F_k(r^*))$ est directement le cipher.
\end{enumerate}
et le premier on remplace par une fonction aléatoire. (étant donné $r\in\{0,1\}^n$ on choisit $f(r):=r'\in\{0,1\}^n$, $2^{n2^n}$)


Au final on écrit 
\begin{defn}
    Pour tout distingueur avec oracle, le distingueur peut pas distinguer la FPA d'une fonction aléatoire.
\end{defn}

\begin{defn}
    De meme on suppose qu'on peut calculer l'inverse de la FPA. Alors F est une permutation pseudo aléatoire forte.
\end{defn}

\subsection{chiffrement par flot}
\begin{defn}
    Paire de DPT :\begin{enumerate}
        \item Init(s, IV), IV est connu et s est la sécurité. Renvoie un état $\eta$.
        \item Nextbit($\eta$) renvoie $\eta'$ et un bit $b$. 
    \end{enumerate}
\end{defn}

Le chiffrement par flot est sécurisé? $JeuSC_{\A,\Pi}(\lambda, b)$ :
\begin{enumerate}
    \item Choix de la seed $s\in \{0,1\}^{\lambda}$
    \item $IV\leftarrow \A(1^{\lambda})$
    \item $\eta\leftarrow Init(s,IV)$
    \item $b'\leftarrow \A^{\Or}$
\end{enumerate}
Ou l'oracle est soit : $\Or(): (b',\eta')\leftarrow Nextbit(\eta)$, $\eta\leftarrow\eta'$, renvoie $b'$. Soit un bit aléatoire.



\section{Fonctions universelles et application}
On definit pour chaque $\lambda\in\N$ : $h^{(\lambda)}~:~K_{\lambda}\times M_{\lambda}\rightarrow T_{\lambda}$
avec $(T_{\lambda},+)$ un groupe. Soit $\epsilon\in [0,1]^{\N}$.

\begin{defn}
    On dit que $h=(h^{(\lambda)})$ est une fonction universelle $\epsilon-$differentiable ssi 
    $\forall\lambda$ $\forall m_1,m_2$, $t\in M_{\lambda}^2\times T_{\lambda}$ :
    $$Pr(h^{(\lambda)}(k,m_1)-h^{(\lambda)}(k,m_2)=t)\leq \epsilon(\lambda)$$
    Ou la proba est prise sur $k\in K_{\lambda}$.
\end{defn}

Soit $\l\in\N$ et $(\mathbb{F}_{\lambda})_{\lambda}$ une famille de corps de grande caractéristique. On def
$h^{(\lambda)}(k\in\mathbb{F}_{\lambda}, (m_1,...,m_{l'-1})\in\mathbb{F}_{\lambda}^{l'-1})
=\sum_{i=1}^{l'-1}m_i k^{l'-i+1}+(l'-1)k$ avec $l'\leq l\leq car(\mathbb{F}_{\lambda})$.
Faut mq $h$ est ($l/\lambda)$-differentiable (assez clair, d'ailleurs le ($l'-1$) est la au cas ou on a deux messages
$(0)^{l'}$, $(0)^{l''}$ et $l'\ne l''$).\\

Soit $f$ une fpa et $h$ une f-u $\epsilon-$diff avec $\epsilon=negl(\lambda)$. On pose \begin{itemize}
    \item $\Pi_{MACuniv}(1^{\lambda}).KeyGen()$ :\begin{enumerate}
                                                        \item $k_h\leftarrow K_{\lambda}$
                                                        \item $k_{fpa}\leftarrow \{0,1\}^{\lambda}$
    \end{enumerate}
    renvoie ($k_h,k_{fpa}$).
    \item MAC($k,m$): \begin{itemize}
        \item $r\leftarrow \{0,1\}^{\lambda}$
        \item renvoie $(r,h(k_h,m)\oplus f(k_{fpa}, r))$
    \end{itemize}
    \item $Verif_{k_h,k_{fpa}}(m,(t_1,t_2))$ : renvoie $h_{k_h}(m)\oplus f_{k_{fpa}(t_1)==t_2}$.
\end{itemize}

Jeu :
$Inforg_{\A,\Pi_{..}}(\lambda)$:
\begin{enumerate}
    \item $k_f\leftarrow\{0,1\}^{\lambda}$
    \item $k_h\leftarrow K_{lambda}$
    \item $Q=\emptyset$
    \item $(m^*,t^*)\leftarrow\A^{\Or_{Mac}}$
    \item renvoie $m^*\notin Q$, $t_2^*=F_k(t_1^*)+h_k(m^*)$
\end{enumerate}
Avec $\Or_{Mac}(m)$:
\begin{itemize}
    \item $Q\leftarrow\{m\}\cup Q$
    \item renvoie (r,$F_{k_fpa}(r)+h_{k_h}(m)$)
\end{itemize}

$Inforg_{\A}^{alea}(\lambda)$: (On peut mettre le choix de $k_h$ APRES $m\notin Q$)
\begin{itemize}
    \item $k_h\leftarrow K_{lambda}$
    \item $Q=\emptyset$
    \item $R=\emptyset$
    \item ...
    \item si $m^*\notin Q$
    \item si $(t, x_t, m_2)\in R$
    \item On renvoie $t_2^*=h_k(m^*)+x_t-h_k(m_2)$
    \item sinon on abandonne
\end{itemize}
avec $\Or_{Mac}$:
\begin{itemize}
    \item $Q=\{m\}\cup Q$
    \item $r\leftarrow\{0,1\}^{\lambda}$
    \item Si $r\in R$ on annule tout
    \item sinon $x\leftarrow \mathbb{F}_q$
    \item $R=R\cup\{(r,x,m)\}$
    \item renvoie $(r,x)$.
\end{itemize}


\newpage
\section{Fonctions de Hashage}

\begin{defn}
    Etant donné $n\in\N$ on définit un couple:
    \begin{enumerate}
        \item ($Gen(1^{\lambda}),~Hash:\{0,1\}^{\lambda}
        \times \{0,1\}^*
        \rightarrow \{0,1\}^n$)
    \end{enumerate}
    Qu'on appelle fonction de Hashage.
\end{defn}

$Coll_H(\lambda):$
    \begin{align*}
        &-s\leftarrow Gen(1^{\lambda})\\
        &-(m_0,m_1)\leftarrow \A(s)\\
        &-renvoie (H^{(s)}(m)==H^{(s)}(m')~\wedge~ m\ne m')\\
    \end{align*}

\begin{defn}
    On dit que $H$ est resistante aux collisions si $\forall$
     PPT $\A$: \[P(Coll_{H,\A}=1)=negl(\lambda)\]
\end{defn}

\subsection{Merkle-Daingard}
Soit $(Gen, h)$ une fonction de compression (Comme une 
fonction de hashage, mais qui prend des messages 
de taille $n' > n$ (la taille de la sortie)).\\
\begin{defn}
    Soit $IV\in\{0,1\}^n$, on pose pour $m_i$ de taille 
    $n'-n$: \[H^s(m_1\ldots m_L)
    =h^{(s)}(h^{(s)}\ldots (h^{(s)}(h^{(s)}(IV||m_1)||m_2)\ldots||m_L)|| L)\]
\end{defn}

La sécurité venant de $h$ se prouve par l'absurde, faut juste 
l'écrire.

\section{Hash and MAC}
Soit $\Pi_{MAX}$ un MAC inforgeable et $H$ une fonction 
de hashage résistante aux collisions.
\begin{defn}
    On définit $\Pi_{Hash\&Mac}$ ou 
    \begin{align*}
        &Gen:~k\leftarrow \Pi_{MAC}.Gen,~s\leftarrow H.Gen\\
        &MAC_{k,s}(m):~\Pi_{MAC}.MAC(H.Hash(m))\\
    \end{align*}
\end{defn}
Si $H$ est vulnérable, une collision donne une forge.
Si $\Pi_{MAC}$ est vulnérable on trouve une collision.\\
\indent Ca se formalise comme ça, $Coll_{\A}(s):$
\begin{align*}
    &k\leftarrow Gen(\lambda)\\    
    &(m^*,t^*)\leftarrow \A^{\Or_{Mac}}()\\
    &si~\exists\in Q~tq~H(m)=H(m^*)~renvoie~(m,m^*)\\
    &sinon~renvoie~
\end{align*}
avec $\Or_{Mac}(m):$
\begin{align*}
    &Q=Q\cup\{m\}\\
    &h\leftarrow H^s(m)\\
    &renvoie~(Mac_k(h))\\
\end{align*}
De l'autre côté $\mathcal{B}^{\A,\Or_{Mac'}}(1^{\lambda}):$
\begin{align*}
    &s\leftarrow Gen(1^{\lambda})\\
    &(m^*,t^*)\leftarrow \A^{\Or_{Mac'}}\\
    &renvoie~(H^s(m^*),t^*)\\
\end{align*}
avec $\Or_{Mac'}(m):$
\begin{align*}
    &renvoie~(\Or_{Mac}(H^s(m)))\\
\end{align*}


\end{document}



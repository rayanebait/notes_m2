\documentclass[12pt]{article}
\usepackage[dvipsnames]{xcolor}
\usepackage{hyperref, pagecolor, mdframed }
\usepackage{array}
\usepackage{tikz}
\usepackage{listings}

\hypersetup{
    colorlinks=true,
    linkcolor=blue,
    urlcolor=red,
    filecolor=RoyalPurple
}

\definecolor{wgrey}{RGB}{148, 38, 55}
\newcommand{\gr}{\color{Sepia}}

\title{reunion rentrée}
\date{19 septembre 2023}
\begin{document}
\maketitle

Mails : fouquet@math.univ-paris-diderotfr, ylg@iriffr 

\section{C : Rappels}
Recuperer le return du main : echo \$?. -Wpointer-arith.\\
gcc -Wall -c test.c\\
gcc -Wall -Wpointer-arith test.c -o test


\subsection{Noms/Pointeurs de fonctions}
On a 
\begin{itemize}
    \item \_\_func\_\_ pour avoir le nom de la function
    \item une fonction (le nom) est un pointeur constant sur le début de ses instructions avec un type 
    \item type (*)(type1, type2, type3,...)
    \item 
\end{itemize}

\subsection{qsort}
Dans \#include$<$stdlib.h$>$ : qsort prend
\begin{itemize}
    \item un void* pour le tableau
    \item un size\_t pour le nb d'elements du tableau
    \item un size\_t pour la taille des elts
    \item int (*compar)(const void *, const void *) la fonction qui permet de comparer deux elt
\end{itemize} 





\end{document}
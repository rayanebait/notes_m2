\documentclass[12pt]{article}
\usepackage[dvipsnames]{xcolor}
\usepackage{hyperref, pagecolor, mdframed }
\usepackage{tabularx, graphicx, amsmath, latexsym, amsfonts, amssymb, amsthm,
amscd, geometry, xspace, enumerate, mathtools}
\usepackage{tikz}

\theoremstyle{plain}
\newtheorem{thm}[subsubsection]{Th\'eor\`eme}
\newtheorem{lem}[subsubsection]{Lemme}
\newtheorem{cor}[subsubsection]{Corollaire}

\theoremstyle{definition}
\newtheorem{defn}[subsubsection]{Definition}
\newtheorem{prop}[subsubsection]{Proposition}

\theoremstyle{remark}
\newtheorem{rem}[subsection]{remarque}

\newcommand{\fdiv}{\textrm{div}}
\newcommand{\Z}{\mathbb{Z}}
\newcommand{\N}{\mathbb{N}}
\newcommand{\Q}{\mathbb{Q}}
\newcommand{\F}{\mathbb{F}}
\newcommand{\algK}{\overline{K}}
\newcommand{\algF}{\overline{\mathbb{F}}}
\newcommand{\Pic}{\textrm{Pic}}
\newcommand{\Hom}{\textrm{Hom}}
\newcommand{\End}{\textrm{End}}
\newcommand{\Disc}{\textrm{Disc}}
\newcommand{\Det}{\textrm{Det}}
\newcommand{\Tr}{\textrm{Tr}}
\newcommand{\Or}{\mathcal{O}}
\newcommand{\OK}{\mathcal{O}_{K}}
\newcommand{\OL}{\mathcal{O}_{L}}
\newcommand{\C}{\mathbb{C}}
\newcommand{\ai}{\mathfrak{a}}
\newcommand{\bi}{\mathfrak{b}}
\newcommand{\w}{\omega}
\newcommand{\gr}{\color{Sepia}}
\newcommand{\rg}{\color{Red}}
\hypersetup{
    colorlinks=true,
    linkcolor=blue,
    urlcolor=Green,
    filecolor=RoyalPurple
}

\newcolumntype{M}[1]{>{\raggedright}m{#1}}
\definecolor{wgrey}{RGB}{148, 38, 55}


\begin{document}
\title{Cryptographie asymétrique}
\date{19 septembre 2023}
\maketitle

\section{CTL}
syntaxe:
\begin{itemize}
    \item $E$ il existe un chemin..
    \item $A$ pour tout chemin..
    \item $\mu$ jusqu'à..
    \item $X$ a la prochaine étape..
    \item $F$ chemin ou a un moment.. 
    \item $G$ chemin ou on a toujours.. 
    \item $\wedge$, $\&$.

\end{itemize}
structure de Kripke:
\begin{defn}
    $\S=(S,s_o,\to,l)$, ou $S$ est un ensemble fini d'état, $\to\subset S\times S$,
    $l:S\to2^{Al}$: associe un ens de prop atomiques à tout état de $\S$.
\end{defn}
\begin{rem}
    Plusieurs anomalies, sans successeurs $AX\phi\equiv T$. 
    Si $\to$ est ``totale'', $AX\phi=\neg tX\neg\phi$ sinon $\neg EX\phi=AX\neg\phi$.
\end{rem}
\subsection{model-checking}
Question, étant donné une structure de Kripke $\S$ et une formule $\phi$.
Est-ce qu'il existe un algorithme qui renvoie $\S,s_0\vDash \phi$. 
Oui ce qu'on fait c'est qu'on découpe la formule en sous formule puis récursion, 
et on vérifie les formules atomiques. On marque chaque sous formules puis on monte 
petit à petit.

\textbf{Algorithme, Cas $\phi=A\phi_1\mu\phi_2$:} 
\begin{itemize}
    \item Marquage($\phi_1$)
    \item Marquage($\phi_2$)
    \item Pour tout $s\in S$:
    \begin{itemize}
        \item $s.\phi:=false$
        \item $s.nbsucc:=deg(s)$(on est sur un graphe)
        \item si $s.\phi_2=T$ alors $L=L\cup\{s\}$
    \end{itemize}
    \item Tant que $L\ne\emptyset$:
    \begin{itemize}
        \item Piocher $s$ dans $L$
        \item $s.\phi:=T$
        \item Pour tout $s'\to s$:
        \begin{itemize}
            \item $s'.nbsucc-=1$
            \item si $s'.nbsucc=0\and s'.\phi_1=T\and s'.\phi_2\ne T$: $L:=L\cup\{s'\}$
        \end{itemize}
    \end{itemize}
\end{itemize}

\begin{prop}
    Décider si $\phi\in CTL$ est vraie pour $\S$ se fait en temps 
    $\Or(|\phi||\S|)$, ($|\S|=|S|+|\to|$). (polynomial)
\end{prop}

Le model checking de LTL est un pb PSPACE-complet ($2^{|\phi|}|\S|$).

\begin{rem}
    $A\phi_1\mu\phi_2\equiv AF\phi_2\and\neg E(\neg\phi_2)\mu(\neg\phi_1\and\neg\phi_2)$ veut 
    simplement dire, on peut pas atteindre $\phi_2$ en croisant un état ou on a ni 
    $\phi_1$ ni $\phi_2$.
\end{rem}

\section{PCTL}
\begin{defn}[Discrete Time Markov Chain]
    Une chaine de Markov: $M=(S,P,s_{init}, l)$ consiste en, $S$ un ensemble d'états (dénombrable),
    $s_{init}$ l'état de départ, $P:S\times S\to [0,1]$ une matrice de probabilités, 
    $l:S\to 2^{Al}$ l'étiquetage des états des props atomiques.
\end{defn}

Si $M$ est finie (i.e. $S$ est fini), $|M|=|S|+\{(s,s')| P(s,s')>0\}$. 

\begin{defn}
    Une chaine de Markov $M$ induit une structure de Kripke $K_{M}=(S, s_{init}, \to,l)$ par 
    $(s,s')\in\to\Leftrightarrow P(s,s')>0$.
\end{defn}

\ldots defs a rajouter 

\subsection{Probabilités}
\begin{defn}[Tribu, $\sigma$-algèbre sur $@W$]
    Ensemble de partie stable par complémentaire, union dénombrable et contenant le vide.
\end{defn}

\begin{defn}[mesure de Probabilité]
    Mesure $\mu$ tq $\mu(@W)=1$.
\end{defn}

\begin{defn}
    On définit $Path^F(M)$ les chemins finis.
\end{defn}

Soit $M=(S,s_0, P, l)$ une chaine de Markov. Soit $\pi_0$ un prefixe de $\pi\in Path(M)$. 

\begin{defn}
    $Cyl(\pi_0):=\{\text{Chemins tq} \pi_0\text{en est un prefixe}\}$.
\end{defn}

Pour nous, $@W$ est l'ens des chemins et $\AA$ la tribu des cylindres de $M$.
\begin{defn}
    La mesure de probabilité sur $\AA$ est déf par la proba sur le préfixe.(produit des transitions)
\end{defn}

\subsection{Propriétés d'accessibilité}
$M$ une chaine de Markov et $A,B\subset S$ des ensembles d'états.

\begin{itemize}
    \item 3 propriétés d'accessibilités:
    \begin{itemize}
        \item $FB=\{\text{chemin qui croise eventuellement B}\}$
        \item $A\mu B=\{\text{chemin dans A jusqu'a croiser B, + croise B eventuellement}\} $
        \item $GFB=\{\text{croise B une infinité de fois}\}$.
    \end{itemize}
\end{itemize}
Etant donné $\phi$ d'un des types décrits avant. 
\[
    P(s\vDash \phi) = P(\{\pi\in Path(M,s)|\pi\vDash\phi\})
\]
Faut vérifier que c'est mesurable:
\begin{itemize}
    \item Pour $FB$ on prend l'union dénombrable des chemins ayant leur bout dans $B$.
    \item Pour $A\mu B$, pareil que $FB$ mais ou le chemin est d'abord dans $A$.
    \item Pour $GFB$ on prend l'intersection de $FB$ et $A\mu B$: 
    \[
        \cap_n\cup_{m\geq n}\cup_{s_n\in B}Cyl(s_0\ldots s_n)
    \]
\end{itemize}

\subsection{Propriétés d'accessibilité}
Pour $s\in S$, on déf $x_s=P(s\vDash FB)$:
\begin{itemize}
    \item $s\in B$, $x_s=1$.
    \item $s\nvDash EFB$, alors $x_s=0$. (exprimable en CTL)
    \item Pour les autres $s\in S_{?}:= \{s\in S| s\notin B \wedge s\vDash EFB\}$:
    \[
        x_s = \sum_{t\in B}p(s,t)+ \sum_{t\in S_{?}}p(s,t)x_t
    \]
\end{itemize}

\noindent Si $\overline{x} = (x_s)_{s\in S_?}\to \overline{x}=\overline{b}+M\overline{x}$. ($M=(p(s,t))_{s,t}$)\\

\noindent On déf aussi $x_s=Pr(s\vDash A\mu B)$:
\begin{itemize}
    \item $s\in B$, $x_s=1$.(noté $S_{=1}\subseteq \{Pr(s\vDash A\mu B)=1\}$, pas d'égalité)
    \item $s\nvDash \to E(A\mu B)$ (il existe un etat qui atteint B en restand dans A),
     alors $x_s=0$. (noté $S_{=0}:=\{s\in S| Pr(s\vDash A\mu B)=0\}$, egalité ici, permet de pas considérer
     les probas)
    \item $S_?=S-(S_{=0}\cup S_{=1})$
\end{itemize}
Soit $\overline{x}=(x_s)_{s\in S_?}$.
\begin{prop}
    $\overline{x}$ est la solution du système d'équations $\overline{y}=M\overline{y}+\overline{b}$
    avec $M$ carrée. ($\overline{b}=(b_s)_{s\in S_?}$ et $b_s=\sum_{t\in B} p(s,t)$)
\end{prop}
(\textbf{On résoud $M\overline{x}=\overline{x}$, clair+unicité.}) \\

On peut aussi caractériser par points fixes. On regarde: 
\[
    \Gamma: [0,1]^{S_?}\to[0,1]^{S_?}
\]
\[
    \Gamma(\overline{y}=M\overline{y}+\overline{b})   
\]
alors $\overline{x}=(x_s)$ avec $x_s=Pr(s\vDash A\mu B)$ est 
le plus petit point fixe de $\Gamma$. On a 
\[
    \Gamma^{n}(x_s)=Pr(s\vDash A\mu^{\leq n}B)
\]
avec $s\vDash EA\mu^{\leq n}B\equiv$ il existe un chemin depuis $s$, $\pi$, tq 
$\exists i\leq n$, $\pi(i)\in B$ et $\forall0\leq j<i$, $\pi(j)\in A$. En gros 
on arrive dans $B$ avant $n$ étapes. Si on pose 
\[
    x_s^{(n)}=Pr(s\vDash A\mu^{\leq n}S_{=1})   
\]
et on a 
\[
    \overline{x}^{(0)}\leq\ldots\leq\overline{x}^{(i)}\leq\ldots\leq \overline{x}
\]
(pour $x\leq y$ si $\forall i,~x_i\leq y_i$)
On prouve 
\[
    x_s^{(n)}=Pr(s\vDash A\mu^{\leq n}S_{=1})   
\]
\begin{itemize}
    \item récurrence: $x_s^{(n+1)}=\sum_{(s,t)\in S_?} p(s,t)x_t^{(n)}+\sum_{t\in S_{=1}} p(s,t)$
    \item le premier terme est en degré $n$ et l'autre $1$.
\end{itemize}
Et on prouve $\overline{x}$ est un point fixe, et le plus petit. 
\begin{itemize}
    \item $x_s=\sum_{t\in S_{=0}} p(s,t)x_t+\sum_{t\in S_{=1}}p(s,t)x_t+\sum_{t\in S_?}p(s,t)x_t$
\end{itemize}

\end{document}
\documentclass[12pt]{article}
\usepackage[dvipsnames]{xcolor}
\usepackage{hyperref, pagecolor, mdframed }
\usepackage{tabularx, graphicx, amsmath, latexsym, amsfonts, amssymb, amsthm,
amscd, geometry, xspace, enumerate, mathtools}
\usepackage{tikz}

\theoremstyle{plain}
\newtheorem{thm}[subsubsection]{Th\'eor\`eme}
\newtheorem{lem}[subsubsection]{Lemme}
\newtheorem{cor}[subsubsection]{Corollaire}

\theoremstyle{definition}
\newtheorem{defn}[subsubsection]{Definition}
\newtheorem{prop}[subsubsection]{Proposition}

\theoremstyle{remark}
\newtheorem{rem}[subsection]{remarque}

\newcommand{\fdiv}{\textrm{div}}
\newcommand{\Z}{\mathbb{Z}}
\newcommand{\N}{\mathbb{N}}
\newcommand{\Q}{\mathbb{Q}}
\newcommand{\F}{\mathbb{F}}
\newcommand{\algK}{\overline{K}}
\newcommand{\algF}{\overline{\mathbb{F}}}
\newcommand{\Pic}{\textrm{Pic}}
\newcommand{\Hom}{\textrm{Hom}}
\newcommand{\End}{\textrm{End}}
\newcommand{\Disc}{\textrm{Disc}}
\newcommand{\Det}{\textrm{Det}}
\newcommand{\Tr}{\textrm{Tr}}
\newcommand{\Or}{\mathcal{O}}
\newcommand{\OK}{\mathcal{O}_{K}}
\newcommand{\OL}{\mathcal{O}_{L}}
\newcommand{\C}{\mathbb{C}}
\newcommand{\ai}{\mathfrak{a}}
\newcommand{\bi}{\mathfrak{b}}
\newcommand{\w}{\omega}
\newcommand{\gr}{\color{Sepia}}
\newcommand{\rg}{\color{Red}}
\hypersetup{
    colorlinks=true,
    linkcolor=blue,
    urlcolor=Green,
    filecolor=RoyalPurple
}

\newcolumntype{M}[1]{>{\raggedright}m{#1}}
\definecolor{wgrey}{RGB}{148, 38, 55}


\begin{document}
\title{Cryptographie asymétrique}
\date{19 septembre 2023}
\maketitle

\section{CTL}
syntaxe:
\begin{itemize}
    \item $E$ il existe un chemin..
    \item $A$ pour tout chemin..
    \item $\mu$ jusqu'à..
    \item $X$ a la prochaine étape..
    \item $F$ chemin ou a un moment.. 
    \item $G$ chemin ou on a toujours.. 
    \item $\wedge$, $\&$.

\end{itemize}
structure de Kripke:
\begin{defn}
    $\S=(S,s_o,\to,l)$, ou $S$ est un ensemble fini d'état, $\to\subset S\times S$,
    $l:S\to2^{Al}$: associe un ens de prop atomiques à tout état de $\S$.
\end{defn}
\begin{rem}
    Plusieurs anomalies, sans successeurs $AX\phi\equiv T$. 
    Si $\to$ est ``totale'', $AX\phi=\neg tX\neg\phi$ sinon $\neg EX\phi=AX\neg\phi$.
\end{rem}
\subsection{model-checking}
Question, étant donné une structure de Kripke $\S$ et une formule $\phi$.
Est-ce qu'il existe un algorithme qui renvoie $\S,s_0\vDash \phi$. 
Oui ce qu'on fait c'est qu'on découpe la formule en sous formule puis récursion, 
et on vérifie les formules atomiques. On marque chaque sous formules puis on monte 
petit à petit.

\textbf{Algorithme, Cas $\phi=A\phi_1\mu\phi_2$:} 
\begin{itemize}
    \item Marquage($\phi_1$)
    \item Marquage($\phi_2$)
    \item Pour tout $s\in S$:
    \begin{itemize}
        \item $s.\phi:=false$
        \item $s.nbsucc:=deg(s)$(on est sur un graphe)
        \item si $s.\phi_2=T$ alors $L=L\cup\{s\}$
    \end{itemize}
    \item Tant que $L\ne\emptyset$:
    \begin{itemize}
        \item Piocher $s$ dans $L$
        \item $s.\phi:=T$
        \item Pour tout $s'\to s$:
        \begin{itemize}
            \item $s'.nbsucc-=1$
            \item si $s'.nbsucc=0\and s'.\phi_1=T\and s'.\phi_2\ne T$: $L:=L\cup\{s'\}$
        \end{itemize}
    \end{itemize}
\end{itemize}

\begin{prop}
    Décider si $\phi\in CTL$ est vraie pour $\S$ se fait en temps 
    $\Or(|\phi||\S|)$, ($|\S|=|S|+|\to|$). (polynomial)
\end{prop}

Le model checking de LTL est un pb PSPACE-complet ($2^{|\phi|}|\S|$).

\begin{rem}
    $A\phi_1\mu\phi_2\equiv AF\phi_2\and\neg E(\neg\phi_2)\mu(\neg\phi_1\and\neg\phi_2)$ veut 
    simplement dire, on peut pas atteindre $\phi_2$ en croisant un état ou on a ni 
    $\phi_1$ ni $\phi_2$.
\end{rem}

\section{PCTL}
\begin{defn}[Discrete Time Markov Chain]
    Une chaine de Markov: $M=(S,P,s_{init}, l)$ consiste en, $S$ un ensemble d'états (dénombrable),
    $s_{init}$ l'état de départ, $P:S\times S\to [0,1]$ une matrice de probabilités, 
    $l:S\to 2^{Al}$ l'étiquetage des états des props atomiques.
\end{defn}

Si $M$ est finie (i.e. $S$ est fini), $|M|=|S|+\{(s,s')| P(s,s')>0\}$. 

\begin{defn}
    Une chaine de Markov $M$ induit une structure de Kripke $K_{M}=(S, s_{init}, \to,l)$ par 
    $(s,s')\in\to\Leftrightarrow P(s,s')>0$.
\end{defn}

\ldots defs a rajouter 

\subsection{Probabilités}
\begin{defn}[Tribu, $\sigma$-algèbre sur $@W$]
    Ensemble de partie stable par complémentaire, union dénombrable et contenant le vide.
\end{defn}

\begin{defn}[mesure de Probabilité]
    Mesure $\mu$ tq $\mu(@W)=1$.
\end{defn}

\begin{defn}
    On définit $Path^F(M)$ les chemins finis.
\end{defn}

Soit $M=(S,s_0, P, l)$ une chaine de Markov. Soit $\pi_0$ un prefixe de $\pi\in Path(M)$. 

\begin{defn}
    $Cyl(\pi_0):=\{\text{Chemins tq} \pi_0\text{en est un prefixe}\}$.
\end{defn}

Pour nous, $@W$ est l'ens des chemins et $\AA$ la tribu des cylindres de $M$.
\begin{defn}
    La mesure de probabilité sur $\AA$ est déf par la proba sur le préfixe.(produit des transitions)
\end{defn}

\subsection{Propriétés d'accessibilité}
$M$ une chaine de Markov et $A,B\subset S$ des ensembles d'états.

\begin{itemize}
    \item 3 propriétés d'accessibilités:
    \begin{itemize}
        \item $FB=\{\text{chemin qui croise eventuellement B}\}$
        \item $A\mu B=\{\text{chemin dans A jusqu'a croiser B, + croise B eventuellement}\} $
        \item $GFB=\{\text{croise B une infinité de fois}\}$.
    \end{itemize}
\end{itemize}
Etant donné $\phi$ d'un des types décrits avant. 
\[
    P(s\vDash \phi) = P(\{\pi\in Path(M,s)|\pi\vDash\phi\})
\]
Faut vérifier que c'est mesurable:
\begin{itemize}
    \item Pour $FB$ on prend l'union dénombrable des chemins ayant leur bout dans $B$.
    \item Pour $A\mu B$, pareil que $FB$ mais ou le chemin est d'abord dans $A$.
    \item Pour $GFB$ on prend l'intersection de $FB$ et $A\mu B$: 
    \[
        \cap_n\cup_{m\geq n}\cup_{s_n\in B}Cyl(s_0\ldots s_n)
    \]
\end{itemize}

\subsection{Propriétés d'accessibilité}
Pour $s\in S$, \textbf{on déf $x_s=P(s\vDash FB)$}:
\begin{itemize}
    \item $s\in B$, $x_s=1$.
    \item $s\nvDash EFB$, alors $x_s=0$. (exprimable en CTL)
    \item Pour les autres $s\in S_{?}:= \{s\in S| s\notin B \wedge s\vDash EFB\}$:
    \[
        x_s = \sum_{t\in B}p(s,t)+ \sum_{t\in S_{?}}p(s,t)x_t
    \]
\end{itemize}

\noindent Si $\overline{x} = (x_s)_{s\in S_?}\to \overline{x}=\overline{b}+M\overline{x}$. ($M=(p(s,t))_{s,t}$)\\

\noindent \textbf{On déf aussi $x_s=Pr(s\vDash A\mu B)$}:
\begin{itemize}
    \item $s\in B$, $x_s=1$.(noté $S_{=1}\subseteq \{Pr(s\vDash A\mu B)=1\}$, pas d'égalité)
    \item $s\nvDash \to E(A\mu B)$ (il existe un etat qui atteint B en restand dans A),
     alors $x_s=0$. (noté $S_{=0}:=\{s\in S| Pr(s\vDash A\mu B)=0\}$, egalité ici, permet de pas considérer
     les probas)
    \item $S_?=S-(S_{=0}\cup S_{=1})$
\end{itemize}
Soit $\overline{x}=(x_s)_{s\in S_?}$.
\begin{prop}
    $\overline{x}$ est la solution du système d'équations $\overline{y}=M\overline{y}+\overline{b}$
    avec $M$ carrée. ($\overline{b}=(b_s)_{s\in S_?}$ et $b_s=\sum_{t\in B} p(s,t)$)
\end{prop}
(\textbf{On résoud $M\overline{x}=\overline{x}$, clair+unicité.}) \\

On peut aussi caractériser par points fixes. On regarde: 
\[
    \Gamma: [0,1]^{S_?}\to[0,1]^{S_?}
\]
\[
    \Gamma(\overline{y}=M\overline{y}+\overline{b})   
\]
alors $\overline{x}=(x_s)$ avec $x_s=Pr(s\vDash A\mu B)$ est 
le plus petit point fixe de $\Gamma$. On a 
\[
    \Gamma^{n}(x_s)=Pr(s\vDash A\mu^{\leq n}B)
\]
avec $s\vDash EA\mu^{\leq n}B\equiv$ il existe un chemin depuis $s$, $\pi$, tq 
$\exists i\leq n$, $\pi(i)\in B$ et $\forall0\leq j<i$, $\pi(j)\in A$. En gros 
on arrive dans $B$ avant $n$ étapes. Si on pose 
\[
    x_s^{(n)}=Pr(s\vDash A\mu^{\leq n}S_{=1})   
\]
et on a 
\[
    \overline{x}^{(0)}\leq\ldots\leq\overline{x}^{(i)}\leq\ldots\leq \overline{x}
\]
(pour $x\leq y$ si $\forall i,~x_i\leq y_i$)
On prouve 
\[
    x_s^{(n)}=Pr(s\vDash A\mu^{\leq n}S_{=1})   
\]
\begin{itemize}
    \item récurrence: $x_s^{(n+1)}=\sum_{(s,t)\in S_?} p(s,t)x_t^{(n)}+\sum_{t\in S_{=1}} p(s,t)$
    \item le premier terme est en degré $n$ et l'autre $1$.
\end{itemize}
Et on prouve $\overline{x}$ est un point fixe, et le plus petit. 
\begin{itemize}
    \item $x_s=\sum_{t\in S_{=0}} p(s,t)x_t+\sum_{t\in S_{=1}}p(s,t)x_t+\sum_{t\in S_?}p(s,t)x_t$
\end{itemize}

\noindent \textbf{Enfin on def $x_s=Pr(s\vDash GFB)$}

\begin{defn}
    Un élt $F$ est dit presque sur sous l'hyp d'un evt $D$
    ssi $Pr(D)=Pr(D\cap F)$
\end{defn}

\textbf{Propriété GF:} Pour une chaine de Markov $M$
(possiblement infinie) et $s,t\in S$, alors on :
\[
    Pr(s\vDash GF t)=Pr(s\vDash \bigwedge_{\pi\in Path^F(t)} GF \pi)
\]
(pour tout $\pi$ préfixe fini partant de $t$.)

\textbf{Preuve:} $\pi=ts_1\ldots s_n$ et on note
 $p= \prod_i Pr(s_i, s_{i+1})$. On montre les proba 
\begin{itemize}
    \item $GFt\wedge G\neg \pi$ nulle.
    \item $GFt\wedge FG\neg \pi$ nulle
\end{itemize}
On déf $E_n(\pi)=$"on visite au moins $n$ fois $t$ et pas
$\pi$ avant au moins $n$ étapes. On a 
\[
    Pr(E_n(\pi))\leq (1-p)^n
\]
On pose $E(\pi)=\bigcap E_n(\pi)$, on croise jamais $\pi$.
On a $E_{n+1}(\pi)\subseteq E_{n}(\pi)$ d'ou 
\[
    Pr(E_(\pi))=\lim_{n\to \infty} 
    Pr(E_n(\pi))\leq \lim_{n\to \infty} (1-p)^n=0
\]
On déf mtn $F_n(_pi)=GFt\wedge X^n\neg F\pi$ puis 
$F(\pi)=\bigcup F_n(\pi)$, on a $F_n\subset F_{n+1}$ d'ou:
\[Pr(s\vDash F(\pi))=\lim_{n\to\infty}Pr(s\vDash F_n(\pi))\]
Et on a en fait $Pr(s\vDash F_n(\pi))=\sum_{s'\in S}
Pr(s\vDash X^n s')Pr(s'\vDash E(\pi))=0$. Enfin
\[F:=\bigcup_{\pi} F(\pi)\]

et \[Pr(s\vDash F)\leq Pr(\sum_{\pi} F(\pi))=0\]
d'ou \[Pr(s\vDash GFt)=Pr(s\vDash GFt\wedge \bigwedge_{\pi}
GF\pi)+Pr(s\vDash GFt\wedge \bigwedge \lor_{\pi} FG\neg \pi)\]
et le deuxième terme vaut 0. \qed\\
\newline
Autrement dit on visite infiniment souvent $t$ 
si et seulement si on visite tout les préfixes finis
sortant de $t$ infiniment souvent.

\begin{defn}
    $CFC(M)$ les composantes fortement connexes (i.e. digraphe
     ou on peut accéder achaque point de chaque point.).
\end{defn}
\begin{defn}
    Une cfc est terminale si $Post^*(C)\subseteq C$ i.e.
    pas de chemin sortant. On appelle $CFCT$ l'ens.
\end{defn}
On note $inf(\pi)$ les états de $\pi$ qui apparaissent 
infiniment.

\begin{prop}
    Si $M$ est une chaine finie. Alors 
    \[Pr(\{\pi/\inf(\pi)\in CFCT(M)\})=1\]
\end{prop}
\textbf{Preuve:}$I(C):=\{\pi/\inf(\pi)\in C\}$, 
\[\sum_{C\in CFC(M)}Pr(I(C))=1\]
Soit $C\in CFC(M)$ tq $Pr(I(C))>0$ et $t\in \inf(\pi)$.
On a $Pr(s\vDash GFt)>0$ d'ou $\forall\pi\in Path^F(t)$,
$Pr(passer par \pi)>0$(en fait 1). Si $C$ n'est pas terminale
on peut en sortir, contradictoire avec $\inf(t)=C$.
Tout les $\pi$ doivent rester dans $C\to$terminale. \qed\\
\begin{cor}
    Si $M$ est une CM finie:
    \[Pr(s\vDash GFt)=
    \begin{cases}
        0& t\notin C\subset CFCT(M)\\
        Pr(s\vDash FC)& sinon\\
    \end{cases}\]
\end{cor}

Objectifs: On suppose que M est finie.

Calculer $S_{\sim\alpha}(c\mu B)$, états vérifiant $c\mu b$. On
a \[
    \sim\in\{=,<,>,\leq,\geq\}
\]
\[
    \alpha\in\{0,1\}
\]

$\to$$S_{=0}(c\mu B)$...

$\to S_{=1}(c\mu B)$. ($c,B\subseteq S$, $M(S,P,s_0, l)$)\\

On construit de chaine de Markov $M'$ a partir de $M$ ou
les états de $B\cup(S\backslash C)$. Sont absorbantes. (i.e.
bouclent sur eux meme avec proba 1)\\
$M'=(S,P', s_0, l)$, avec :
\[
    P'(s,t)=\begin{cases}
        1& si~t=s~et~s\in B\cup S\backslash C\\
        0& si~t\ne s~et~s\in B\cup S\backslash C\\
        P'(s,t)& sinon\\
    \end{cases}
\]

Pour les états de $B\cup S\backslash C$ on connait 
leur proba de vérifier $c\mu B$. \[B\to 1\]
\[S\backslash (C\cup B)\to 0\]

On a \[Pr^M(s\vDash c\mu B)=Pr^{M'}(s\vDash FB)\]
ET \[Pr^M(s\vDash c\mu B)=Pr^{M'}(s\vDash FB)=1~si~s\in B\]
\[Pr^M(s\vDash c\mu B)=Pr^{M'}(s\vDash FB)=0~si~s\in S
\backslash(C\cup B)\]

cas général ? Le pb est désormais de calculer 
\[S_{=1}(FB)\] i.e. $\{s|Pr(s\vDash FB)=1\}$.\\
\newpage 
\noindent On a l'\textbf{équivalence} suivante:
\begin{enumerate}
    \item $Pr(s\vDash FB)=1$
    \item $Post^*(t)\cap B\ne \emptyset$, $\forall t\in Post^*(s)$.
    \item $s\in S\backslash Pre^*(S\backslash Pre^*(B))$.
\end{enumerate}
\noindent\textbf{Preuve:} $1.\implies 2.$ est clair. 
$2.\implies 1.$ Une execution depuis $s$ finit avec proba 
$1$ dans une CFCT. Celles ci étant de deux types. 
\begin{enumerate}
    \item Singleton dans B.
    \item Cycle d'états dont aucun est dans B.
\end{enumerate}
(faut se rappeler que $Post^*(C)=C$) Pour tout état d'une
CFCT, on a $t\in Post^*(C)$ donc $Post^*(t)\cap B\ne 0$.
Donc on peut pas avoir une CFCT comme $2.$ donc la proba 
d'avoir $G\neg B$ est nulle. $2. \equiv 3.$ 
\[Post^*(t)\cap B\ne \emptyset~~~\forall t\in Post^*(s)\]
\[\Leftrightarrow Post^*(s)\subseteq Pre^*(B)\]
\[\Leftrightarrow Post^*(s)\cap S\backslash Pre^*(B)=\emptyset\]
\[\Leftrightarrow s\notin Pre^*(S\backslash Pre^*(B))\]
\[\Leftrightarrow s\in S\backslash Pre^*(S\backslash Pre^*(B))\]\qed
\begin{cor}
    Pour calculer $S_{=1}(c\mu B)$ on construit $M'$ 
    puis on calcule $S\backslash Pre^*(S\backslash Pre^*(B))$.
    Temps linéaire en $|M|$.
\end{cor}

Maintenant pour l'accessibilité répétée ? $GF B$?\\
On a:
\begin{enumerate}
    \item $Pr(s\vDash GFB)=1$
    \item $C\cap B\ne \emptyset$ pour toute CFCT $C$
    atteignable depuis $s$.
    \item $s\vDash AG~EF~B$ (CTL).
\end{enumerate}
Pour tous ces ensembles $S_{=1,0}$ on a pas utilisé 
la valeur de la proba, juste $>0$! Y s'avére que 
c'est vrai uniquement parce qu'on regarde des chaines 
de Markov finies.\\ \indent Vérifier les props \textbf{qualitatives}
peut nécessiter de regarder la valeur réelle.

\section{PCTL, 2}
\textbf{Syntaxe}:
\begin{itemize}
    \item $\phi_1,\phi_2:= T| a| \phi_1\wedge \phi_2|\neg \phi_1
    P_J(\phi_l)$, avec $a\in AP$ et $J\subseteq [0,1]$, aux bornes rationelles.
    \item $\phi_l:= X\phi_1|\phi_1\mu \phi_2 | \phi_1 \mu^{\leq n}\phi_2$
\end{itemize}
Ou aussi : $X, F=T\mu \phi$. On utilisera l'opérateur $G$.
En pratique on se limite à $J=[0,1],[0,p[,[p;1], ]p;1]$. (P est 
pas une probabilité.)

\begin{itemize}
    \item Pour $G$: 
    $P_{\leq \alpha}(G\phi)=P_{\geq 1-\alpha}(F\neg\phi)$
    \item $G^{\leq n}\phi=\phi$ est vraie pour les $n+1$
    premier états. 
    \[P_{\leq \alpha}(G^{\leq n}\phi)=
    P_{\geq 1-\alpha}(F^{\leq n}\neg \phi)\]
\end{itemize}
\textbf{Sémantique}:
\begin{itemize}
    \item $M=(S,s_0, P, l)$ une chaine de Markov.
    \item $s\vDash T$ toujours
    \item $s\vDash e$ ssi $e\in l(s)$
    \item $s\vDash \phi_1\wedge \phi_2$ ssi ($s\vDash \phi_1~et~s\vDash \phi_2$)
    \item $s\vDash \neg \phi_1$ ssi $s\nvDash \phi_1$
    \item $s\vDash P_J(\phi_l)$ ssi $Pr(s\vDash \phi_l)\in J$
    i.e. $Pr\{\phi\in Path(s)|\pi\vDash \phi_l\}$.
    \item $\pi\vDash X\phi_1$ ssi $\pi(1)\vDash \phi_1$
    \item $\pi\vDash \phi_1\mu \phi_2$ ssi $\exists i\geq 0$
    ($\pi(i)\vDash \phi_2 \wedge \forall 0\leq j<i, \pi(j)\vDash \phi_1$)
    \item $\pi\vDash \phi_2\mu^{\leq n}\phi_2$ ssi 
    $\exists 0\leq i\leq n$, $\pi(i)\vDash \phi_2 \wedge 
    (\forall 0\leq j<i,~\pi(j)\vDash \phi_1)$

\end{itemize}

Equivalence de formules: $\forall M$, $\forall s$,
 $M,s\vDash \phi_1\Leftrightarrow M,s\vDash \phi_2$.

\begin{prop}
    $\alpha\in[0,1]$, 
    $P_{<\alpha}(\phi)\equiv \neg P_{\geq \alpha}(\phi)$
\end{prop}

\noindent \textbf{Model Checking}:
Pour $M$ finie, $M\vDash \phi$. On fait comme pour CTL,
on vérifie fait un récursion sur les sous formules.\\
\begin{itemize}
    \item Changements:
    \begin{itemize}
        \item $P_{\sim\alpha}(X\phi)$
        \item $P_{\sim\alpha}(\phi_1\mu\phi2)$
        \item $P_{\sim\alpha(\phi_1\mu^{\leq n}\phi_2)}$
    \end{itemize}
\end{itemize}
On déf $Sat(\phi)$ les états de $M$ qui vérifient $\phi$.
\begin{itemize}
    \item Calcul de: $Sat(P_{\sim\alpha}(X\phi))$
    \item $Pr(s\vDash X\phi)=\sum_{s'\in Sat(\phi)}P(s,s')$
    \item Reste à comparer avec $\sim\alpha$
    \item Calcul de $Sat(P_{\sim\alpha}(\phi_1\mu\phi_2))$
    \item Calculer $Sat(\phi_1,2)$
    \item Construire $M'$ avc les états de $\neg\phi_1\wedge 
    \neg\phi_2$ et $\phi_2$
    \item Reste à calculer les probas d'atteindre 
    $Sat(\phi_2)$ depuis tout état de $M'$. $"FSat(\phi_2)"$
    \item Calcul de $Sat(P_{\sim\alpha}(\phi_1\mu^{\leq n}\phi_2))$ 
    \item Calculer $Sat(\phi_1,2)$
    \item Calculer $M'$ avec les états $\neg\phi_1\wedge \neg \phi_2$ ou 
    $\phi_2$ sont absorants(ou comme union, 
    $Sat(\neg\phi_1\wedge \neg\phi_2)\cup Sat(\phi_2)$). 
    \item $M'=(S,s_0,P',l)$.
    \item Calculer $P'*P'\ldots*P'$ avec $n$ termes.
\end{itemize}

\noindent \textbf{Conclusion}: Le modèle checking est 
en temps $\Or(poly(|M|).|\phi|.n_{max}$), avec 
$n_{max}$ le plus grand $n$ de $\mu^{n}$ apparaissant.




    


\end{document}
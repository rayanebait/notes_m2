\documentclass[12pt]{article}
\usepackage[dvipsnames]{xcolor}
\usepackage{hyperref, pagecolor, mdframed }
\usepackage{tabularx, graphicx, amsmath, latexsym, amsfonts, amssymb, amsthm,
amscd, geometry, xspace, enumerate, mathtools}
\usepackage{tikz}

\theoremstyle{plain}
\newtheorem{thm}[subsubsection]{Th\'eor\`eme}
\newtheorem{lem}[subsubsection]{Lemme}
\newtheorem{cor}[subsubsection]{Corollaire}

\theoremstyle{definition}
\newtheorem{defn}[subsubsection]{Definition}
\newtheorem{prop}[subsubsection]{Proposition}

\theoremstyle{remark}
\newtheorem{rem}[subsection]{remarque}

\newcommand{\fdiv}{\textrm{div}}
\newcommand{\Z}{\mathbb{Z}}
\newcommand{\N}{\mathbb{N}}
\newcommand{\Q}{\mathbb{Q}}
\newcommand{\F}{\mathbb{F}}
\newcommand{\algK}{\overline{K}}
\newcommand{\algF}{\overline{\mathbb{F}}}
\newcommand{\Pic}{\textrm{Pic}}
\newcommand{\Hom}{\textrm{Hom}}
\newcommand{\End}{\textrm{End}}
\newcommand{\Disc}{\textrm{Disc}}
\newcommand{\Det}{\textrm{Det}}
\newcommand{\Tr}{\textrm{Tr}}
\newcommand{\Or}{\mathcal{O}}
\newcommand{\OK}{\mathcal{O}_{K}}
\newcommand{\OL}{\mathcal{O}_{L}}
\newcommand{\C}{\mathbb{C}}
\newcommand{\ai}{\mathfrak{a}}
\newcommand{\bi}{\mathfrak{b}}
\newcommand{\w}{\omega}
\newcommand{\gr}{\color{Sepia}}
\newcommand{\rg}{\color{Red}}
\hypersetup{
    colorlinks=true,
    linkcolor=blue,
    urlcolor=Green,
    filecolor=RoyalPurple
}

\newcolumntype{M}[1]{>{\raggedright}m{#1}}
\definecolor{wgrey}{RGB}{148, 38, 55}


\begin{document}
\title{Cryptographie asymétrique}
\date{19 septembre 2023}
\maketitle

\section{CTL}
syntaxe:
\begin{itemize}
    \item $E$ il existe un chemin..
    \item $A$ pour tout chemin..
    \item $\mu$ jusqu'à..
    \item $X$ a la prochaine étape..
    \item $F$ chemin ou a un moment.. 
    \item $G$ chemin ou on a toujours.. 
    \item $\wedge$, $\&$.

\end{itemize}
structure de Kripke:
\begin{defn}
    $\S=(S,s_o,\to,l)$, ou $S$ est un ensemble fini d'état, $\to\subset S\times S$,
    $l:S\to2^{Al}$: associe un ens de prop atomiques à tout état de $\S$.
\end{defn}
\begin{rem}
    Plusieurs anomalies, sans successeurs $AX\phi\equiv T$. 
    Si $\to$ est ``totale'', $AX\phi=\neg tX\neg\phi$ sinon $\neg EX\phi=AX\neg\phi$.
\end{rem}
\subsection{model-checking}
Question, étant donné une structure de Kripke $\S$ et une formule $\phi$.
Est-ce qu'il existe un algorithme qui renvoie $\S,s_0\vDash \phi$. 
Oui ce qu'on fait c'est qu'on découpe la formule en sous formule puis récursion, 
et on vérifie les formules atomiques. On marque chaque sous formules puis on monte 
petit à petit.

\textbf{Algorithme, Cas $\phi=A\phi_1\mu\phi_2$:} 
\begin{itemize}
    \item Marquage($\phi_1$)
    \item Marquage($\phi_2$)
    \item Pour tout $s\in S$:
    \begin{itemize}
        \item $s.\phi:=false$
        \item $s.nbsucc:=deg(s)$(on est sur un graphe)
        \item si $s.\phi_2=T$ alors $L=L\cup\{s\}$
    \end{itemize}
    \item Tant que $L\ne\emptyset$:
    \begin{itemize}
        \item Piocher $s$ dans $L$
        \item $s.\phi:=T$
        \item Pour tout $s'\to s$:
        \begin{itemize}
            \item $s'.nbsucc-=1$
            \item si $s'.nbsucc=0\and s'.\phi_1=T\and s'.\phi_2\ne T$: $L:=L\cup\{s'\}$
        \end{itemize}
    \end{itemize}
\end{itemize}

\begin{prop}
    Décider si $\phi\in CTL$ est vraie pour $\S$ se fait en temps 
    $\Or(|\phi||\S|)$, ($|\S|=|S|+|\to|$). (polynomial)
\end{prop}

Le model checking de LTL est un pb PSPACE-complet ($2^{|\phi|}|\S|$).

\begin{rem}
    $A\phi_1\mu\phi_2\equiv AF\phi_2\and\neg E(\neg\phi_2)\mu(\neg\phi_1\and\neg\phi_2)$ veut 
    simplement dire, on peut pas atteindre $\phi_2$ en croisant un état ou on a ni 
    $\phi_1$ ni $\phi_2$.
\end{rem}

\section{PCTL}
\begin{defn}[Discrete Time Markov Chain]
    Une chaine de Markov: $M=(S,P,s_{init}, l)$ consiste en, $S$ un ensemble d'états (dénombrable),
    $s_{init}$ l'état de départ, $P:S\times S\to [0,1]$ une matrice de probabilités, 
    $l:S\to 2^{Al}$ l'étiquetage des états des props atomiques.
\end{defn}

Si $M$ est finie (i.e. $S$ est fini), $|M|=|S|+\{(s,s')| P(s,s')>0\}$. 

\begin{defn}
    Une chaine de Markov $M$ induit une structure de Kripke $K_{M}=(S, s_{init}, \to,l)$ par 
    $(s,s')\in\to\Leftrightarrow P(s,s')>0$.
\end{defn}

\ldots defs a rajouter 

\subsection{Probabilités}
\begin{defn}[Tribu, $\sigma$-algèbre sur $@W$]
    Ensemble de partie stable par complémentaire, union dénombrable et contenant le vide.
\end{defn}

\begin{defn}[mesure de Probabilité]
    Mesure $\mu$ tq $\mu(@W)=1$.
\end{defn}

\begin{defn}
    On définit $Path^F(M)$ les chemins finis.
\end{defn}

Soit $M=(S,s_0, P, l)$ une chaine de Markov. Soit $\pi_0$ un prefixe de $\pi\in Path(M)$. 

\begin{defn}
    $Cyl(\pi_0):=\{\text{Chemins tq} \pi_0\text{en est un prefixe}\}$.
\end{defn}

Pour nous, $@W$ est l'ens des chemins et $\AA$ la tribu des cylindres de $M$.
\begin{defn}
    La mesure de probabilité sur $\AA$ est déf par la proba sur le préfixe.(produit des transitions)
\end{defn}

\subsection{Propriétés d'accessibilité}
$M$ une chaine de Markov et $A,B\subset S$ des ensembles d'états.

\begin{itemize}
    \item 3 propriétés d'accessibilités:
    \begin{itemize}
        \item $FB=\{\text{chemin qui croise eventuellement B}\}$
        \item $A\mu B=\{\text{chemin dans A jusqu'a croiser B, + croise B eventuellement}\} $
        \item $GFB=\{\text{croise B une infinité de fois}\}$.
    \end{itemize}
\end{itemize}
Etant donné $\phi$ d'un des types décrits avant. 
\[
    P(s\vDash \phi) = P(\{\pi\in Path(M,s)|\pi\vDash\phi\})
\]
Faut vérifier que c'est mesurable:
\begin{itemize}
    \item Pour $FB$ on prend l'union dénombrable des chemins ayant leur bout dans $B$.
    \item Pour $A\mu B$, pareil que $FB$ mais ou le chemin est d'abord dans $A$.
    \item Pour $GFB$ on prend l'intersection de $FB$ et $A\mu B$: 
    \[
        \cap_n\cup_{m\geq n}\cup_{s_n\in B}Cyl(s_0\ldots s_n)
    \]
\end{itemize}

\subsection{Propriétés d'accessibilité}
Pour $s\in S$, \textbf{on déf $x_s=P(s\vDash FB)$}:
\begin{itemize}
    \item $s\in B$, $x_s=1$.
    \item $s\nvDash EFB$, alors $x_s=0$. (exprimable en CTL)
    \item Pour les autres $s\in S_{?}:= \{s\in S| s\notin B \wedge s\vDash EFB\}$:
    \[
        x_s = \sum_{t\in B}p(s,t)+ \sum_{t\in S_{?}}p(s,t)x_t
    \]
\end{itemize}

\noindent Si $\overline{x} = (x_s)_{s\in S_?}\to \overline{x}=\overline{b}+M\overline{x}$. ($M=(p(s,t))_{s,t}$)\\

\noindent \textbf{On déf aussi $x_s=Pr(s\vDash A\mu B)$}:
\begin{itemize}
    \item $s\in B$, $x_s=1$.(noté $S_{=1}\subseteq \{Pr(s\vDash A\mu B)=1\}$, pas d'égalité)
    \item $s\nvDash \to E(A\mu B)$ (il existe un etat qui atteint B en restand dans A),
     alors $x_s=0$. (noté $S_{=0}:=\{s\in S| Pr(s\vDash A\mu B)=0\}$, egalité ici, permet de pas considérer
     les probas)
    \item $S_?=S-(S_{=0}\cup S_{=1})$
\end{itemize}
Soit $\overline{x}=(x_s)_{s\in S_?}$.
\begin{prop}
    $\overline{x}$ est la solution du système d'équations $\overline{y}=M\overline{y}+\overline{b}$
    avec $M$ carrée. ($\overline{b}=(b_s)_{s\in S_?}$ et $b_s=\sum_{t\in B} p(s,t)$)
\end{prop}
(\textbf{On résoud $M\overline{x}=\overline{x}$, clair+unicité.}) \\

On peut aussi caractériser par points fixes. On regarde: 
\[
    \Gamma: [0,1]^{S_?}\to[0,1]^{S_?}
\]
\[
    \Gamma(\overline{y}=M\overline{y}+\overline{b})   
\]
alors $\overline{x}=(x_s)$ avec $x_s=Pr(s\vDash A\mu B)$ est 
le plus petit point fixe de $\Gamma$. On a 
\[
    \Gamma^{n}(x_s)=Pr(s\vDash A\mu^{\leq n}B)
\]
avec $s\vDash EA\mu^{\leq n}B\equiv$ il existe un chemin depuis $s$, $\pi$, tq 
$\exists i\leq n$, $\pi(i)\in B$ et $\forall0\leq j<i$, $\pi(j)\in A$. En gros 
on arrive dans $B$ avant $n$ étapes. Si on pose 
\[
    x_s^{(n)}=Pr(s\vDash A\mu^{\leq n}S_{=1})   
\]
et on a 
\[
    \overline{x}^{(0)}\leq\ldots\leq\overline{x}^{(i)}\leq\ldots\leq \overline{x}
\]
(pour $x\leq y$ si $\forall i,~x_i\leq y_i$)
On prouve 
\[
    x_s^{(n)}=Pr(s\vDash A\mu^{\leq n}S_{=1})   
\]
\begin{itemize}
    \item récurrence: $x_s^{(n+1)}=\sum_{(s,t)\in S_?} p(s,t)x_t^{(n)}+\sum_{t\in S_{=1}} p(s,t)$
    \item le premier terme est en degré $n$ et l'autre $1$.
\end{itemize}
Et on prouve $\overline{x}$ est un point fixe, et le plus petit. 
\begin{itemize}
    \item $x_s=\sum_{t\in S_{=0}} p(s,t)x_t+\sum_{t\in S_{=1}}p(s,t)x_t+\sum_{t\in S_?}p(s,t)x_t$
\end{itemize}

\noindent \textbf{Enfin on def $x_s=Pr(s\vDash GFB)$}

\begin{defn}
    Un élt $F$ est dit presque sur sous l'hyp d'un evt $D$
    ssi $Pr(D)=Pr(D\cap F)$
\end{defn}

\textbf{Propriété GF:} Pour une chaine de Markov $M$
(possiblement infinie) et $s,t\in S$, alors on :
\[
    Pr(s\vDash GF t)=Pr(s\vDash \bigwedge_{\pi\in Path^F(t)} GF \pi)
\]
(pour tout $\pi$ préfixe fini partant de $t$.)

\textbf{Preuve:} $\pi=ts_1\ldots s_n$ et on note
 $p= \prod_i Pr(s_i, s_{i+1})$. On montre les proba 
\begin{itemize}
    \item $GFt\wedge G\neg \pi$ nulle.
    \item $GFt\wedge FG\neg \pi$ nulle
\end{itemize}
On déf $E_n(\pi)=$"on visite au moins $n$ fois $t$ et pas
$\pi$ avant au moins $n$ étapes. On a 
\[
    Pr(E_n(\pi))\leq (1-p)^n
\]
On pose $E(\pi)=\bigcap E_n(\pi)$, on croise jamais $\pi$.
On a $E_{n+1}(\pi)\subseteq E_{n}(\pi)$ d'ou 
\[
    Pr(E_(\pi))=\lim_{n\to \infty} 
    Pr(E_n(\pi))\leq \lim_{n\to \infty} (1-p)^n=0
\]
On déf mtn $F_n(_pi)=GFt\wedge X^n\neg F\pi$ puis 
$F(\pi)=\bigcup F_n(\pi)$, on a $F_n\subset F_{n+1}$ d'ou:
\[Pr(s\vDash F(\pi))=\lim_{n\to\infty}Pr(s\vDash F_n(\pi))\]
Et on a en fait $Pr(s\vDash F_n(\pi))=\sum_{s'\in S}
Pr(s\vDash X^n s')Pr(s'\vDash E(\pi))=0$. Enfin
\[F:=\bigcup_{\pi} F(\pi)\]

et \[Pr(s\vDash F)\leq Pr(\sum_{\pi} F(\pi))=0\]
d'ou \[Pr(s\vDash GFt)=Pr(s\vDash GFt\wedge \bigwedge_{\pi}
GF\pi)+Pr(s\vDash GFt\wedge \bigwedge \lor_{\pi} FG\neg \pi)\]
et le deuxième terme vaut 0. \qed\\
\newline
Autrement dit on visite infiniment souvent $t$ 
si et seulement si on visite tout les préfixes finis
sortant de $t$ infiniment souvent.

\begin{defn}
    $CFC(M)$ les composantes fortement connexes (i.e. digraphe
     ou on peut accéder achaque point de chaque point.).
\end{defn}
\begin{defn}
    Une cfc est terminale si $Post^*(C)\subseteq C$ i.e.
    pas de chemin sortant. On appelle $CFCT$ l'ens.
\end{defn}
On note $inf(\pi)$ les états de $\pi$ qui apparaissent 
infiniment.

\begin{prop}
    Si $M$ est une chaine finie. Alors 
    \[Pr(\{\pi/\inf(\pi)\in CFCT(M)\})=1\]
\end{prop}
\textbf{Preuve:}$I(C):=\{\pi/\inf(\pi)\in C\}$, 
\[\sum_{C\in CFC(M)}Pr(I(C))=1\]
Soit $C\in CFC(M)$ tq $Pr(I(C))>0$ et $t\in \inf(\pi)$.
On a $Pr(s\vDash GFt)>0$ d'ou $\forall\pi\in Path^F(t)$,
$Pr(passer par \pi)>0$(en fait 1). Si $C$ n'est pas terminale
on peut en sortir, contradictoire avec $\inf(t)=C$.
Tout les $\pi$ doivent rester dans $C\to$terminale. \qed\\
\begin{cor}
    Si $M$ est une CM finie:
    \[Pr(s\vDash GFt)=
    \begin{cases}
        0& t\notin C\subset CFCT(M)\\
        Pr(s\vDash FC)& sinon\\
    \end{cases}\]
\end{cor}

Objectifs: On suppose que M est finie.

Calculer $S_{\sim\alpha}(c\mu B)$, états vérifiant $c\mu b$. On
a \[
    \sim\in\{=,<,>,\leq,\geq\}
\]
\[
    \alpha\in\{0,1\}
\]

$\to$$S_{=0}(c\mu B)$...

$\to S_{=1}(c\mu B)$. ($c,B\subseteq S$, $M(S,P,s_0, l)$)\\

On construit de chaine de Markov $M'$ a partir de $M$ ou
les états de $B\cup(S\backslash C)$. Sont absorbantes. (i.e.
bouclent sur eux meme avec proba 1)\\
$M'=(S,P', s_0, l)$, avec :
\[
    P'(s,t)=\begin{cases}
        1& si~t=s~et~s\in B\cup S\backslash C\\
        0& si~t\ne s~et~s\in B\cup S\backslash C\\
        P'(s,t)& sinon\\
    \end{cases}
\]

Pour les états de $B\cup S\backslash C$ on connait 
leur proba de vérifier $c\mu B$. \[B\to 1\]
\[S\backslash (C\cup B)\to 0\]

On a \[Pr^M(s\vDash c\mu B)=Pr^{M'}(s\vDash FB)\]
ET \[Pr^M(s\vDash c\mu B)=Pr^{M'}(s\vDash FB)=1~si~s\in B\]
\[Pr^M(s\vDash c\mu B)=Pr^{M'}(s\vDash FB)=0~si~s\in S
\backslash(C\cup B)\]

cas général ? Le pb est désormais de calculer 
\[S_{=1}(FB)\] i.e. $\{s|Pr(s\vDash FB)=1\}$.\\
\newpage 
\noindent On a l'\textbf{équivalence} suivante:
\begin{enumerate}
    \item $Pr(s\vDash FB)=1$
    \item $Post^*(t)\cap B\ne \emptyset$, $\forall t\in Post^*(s)$.
    \item $s\in S\backslash Pre^*(S\backslash Pre^*(B))$.
\end{enumerate}
\noindent\textbf{Preuve:} $1.\implies 2.$ est clair. 
$2.\implies 1.$ Une execution depuis $s$ finit avec proba 
$1$ dans une CFCT. Celles ci étant de deux types. 
\begin{enumerate}
    \item Singleton dans B.
    \item Cycle d'états dont aucun est dans B.
\end{enumerate}
(faut se rappeler que $Post^*(C)=C$) Pour tout état d'une
CFCT, on a $t\in Post^*(C)$ donc $Post^*(t)\cap B\ne 0$.
Donc on peut pas avoir une CFCT comme $2.$ donc la proba 
d'avoir $G\neg B$ est nulle. $2. \equiv 3.$ 
\[Post^*(t)\cap B\ne \emptyset~~~\forall t\in Post^*(s)\]
\[\Leftrightarrow Post^*(s)\subseteq Pre^*(B)\]
\[\Leftrightarrow Post^*(s)\cap S\backslash Pre^*(B)=\emptyset\]
\[\Leftrightarrow s\notin Pre^*(S\backslash Pre^*(B))\]
\[\Leftrightarrow s\in S\backslash Pre^*(S\backslash Pre^*(B))\]\qed
\begin{cor}
    Pour calculer $S_{=1}(c\mu B)$ on construit $M'$ 
    puis on calcule $S\backslash Pre^*(S\backslash Pre^*(B))$.
    Temps linéaire en $|M|$.
\end{cor}

Maintenant pour l'accessibilité répétée ? $GF B$?\\
On a:
\begin{enumerate}
    \item $Pr(s\vDash GFB)=1$
    \item $C\cap B\ne \emptyset$ pour toute CFCT $C$
    atteignable depuis $s$.
    \item $s\vDash AG~EF~B$ (CTL).
\end{enumerate}
Pour tous ces ensembles $S_{=1,0}$ on a pas utilisé 
la valeur de la proba, juste $>0$! Y s'avére que 
c'est vrai uniquement parce qu'on regarde des chaines 
de Markov finies.\\ \indent Vérifier les props \textbf{qualitatives}
peut nécessiter de regarder la valeur réelle.

\section{PCTL, 2}
\textbf{Syntaxe}:
\begin{itemize}
    \item $\phi_1,\phi_2:= T| a| \phi_1\wedge \phi_2|\neg \phi_1
    P_J(\phi_l)$, avec $a\in AP$ et $J\subseteq [0,1]$, aux bornes rationelles.
    \item $\phi_l:= X\phi_1|\phi_1\mu \phi_2 | \phi_1 \mu^{\leq n}\phi_2$
\end{itemize}
Ou aussi : $X, F=T\mu \phi$. On utilisera l'opérateur $G$.
En pratique on se limite à $J=[0,1],[0,p[,[p;1], ]p;1]$. (P est 
pas une probabilité.)

\begin{itemize}
    \item Pour $G$: 
    $P_{\leq \alpha}(G\phi)=P_{\geq 1-\alpha}(F\neg\phi)$
    \item $G^{\leq n}\phi=\phi$ est vraie pour les $n+1$
    premier états. 
    \[P_{\leq \alpha}(G^{\leq n}\phi)=
    P_{\geq 1-\alpha}(F^{\leq n}\neg \phi)\]
\end{itemize}
\textbf{Sémantique}:
\begin{itemize}
    \item $M=(S,s_0, P, l)$ une chaine de Markov.
    \item $s\vDash T$ toujours
    \item $s\vDash e$ ssi $e\in l(s)$
    \item $s\vDash \phi_1\wedge \phi_2$ ssi ($s\vDash \phi_1~et~s\vDash \phi_2$)
    \item $s\vDash \neg \phi_1$ ssi $s\nvDash \phi_1$
    \item $s\vDash P_J(\phi_l)$ ssi $Pr(s\vDash \phi_l)\in J$
    i.e. $Pr\{\phi\in Path(s)|\pi\vDash \phi_l\}$.
    \item $\pi\vDash X\phi_1$ ssi $\pi(1)\vDash \phi_1$
    \item $\pi\vDash \phi_1\mu \phi_2$ ssi $\exists i\geq 0$
    ($\pi(i)\vDash \phi_2 \wedge \forall 0\leq j<i, \pi(j)\vDash \phi_1$)
    \item $\pi\vDash \phi_2\mu^{\leq n}\phi_2$ ssi 
    $\exists 0\leq i\leq n$, $\pi(i)\vDash \phi_2 \wedge 
    (\forall 0\leq j<i,~\pi(j)\vDash \phi_1)$

\end{itemize}

Equivalence de formules: $\forall M$, $\forall s$,
 $M,s\vDash \phi_1\Leftrightarrow M,s\vDash \phi_2$.

\begin{prop}
    $\alpha\in[0,1]$, 
    $P_{<\alpha}(\phi)\equiv \neg P_{\geq \alpha}(\phi)$
\end{prop}

\noindent \textbf{Model Checking}:
Pour $M$ finie, $M\vDash \phi$. On fait comme pour CTL,
on vérifie fait un récursion sur les sous formules.\\
\begin{itemize}
    \item Changements:
    \begin{itemize}
        \item $P_{\sim\alpha}(X\phi)$
        \item $P_{\sim\alpha}(\phi_1\mu\phi2)$
        \item $P_{\sim\alpha(\phi_1\mu^{\leq n}\phi_2)}$
    \end{itemize}
\end{itemize}
On déf $Sat(\phi)$ les états de $M$ qui vérifient $\phi$.
\begin{itemize}
    \item Calcul de: $Sat(P_{\sim\alpha}(X\phi))$
    \item $Pr(s\vDash X\phi)=\sum_{s'\in Sat(\phi)}P(s,s')$
    \item Reste à comparer avec $\sim\alpha$
    \item Calcul de $Sat(P_{\sim\alpha}(\phi_1\mu\phi_2))$
    \item Calculer $Sat(\phi_1,2)$
    \item Construire $M'$ avc les états de $\neg\phi_1\wedge 
    \neg\phi_2$ et $\phi_2$
    \item Reste à calculer les probas d'atteindre 
    $Sat(\phi_2)$ depuis tout état de $M'$. $"FSat(\phi_2)"$
    \item Calcul de $Sat(P_{\sim\alpha}(\phi_1\mu^{\leq n}\phi_2))$ 
    \item Calculer $Sat(\phi_1,2)$
    \item Calculer $M'$ avec les états $\neg\phi_1\wedge \neg \phi_2$ ou 
    $\phi_2$ sont absorants(ou comme union, 
    $Sat(\neg\phi_1\wedge \neg\phi_2)\cup Sat(\phi_2)$). 
    \item $M'=(S,s_0,P',l)$.
    \item Calculer $P'*P'\ldots*P'$ avec $n$ termes.
\end{itemize}

\noindent \textbf{Conclusion}: Le modèle checking est 
en temps $\Or(poly(|M|).|\phi|.n_{max}$), avec 
$n_{max}$ le plus grand $n$ de $\mu^{n}$ apparaissant.

\section{PCTL, 0/1}

On considère que les intervalles: $<1,>0, =1,=0$.
On veut comparer PCTL, 0/1 et CTL. Comme dans le td.
\begin{itemize}
    \item En général, incomparables. (y'a des trucs qu'on 
    peut faire dans l'un et pas dans l'autre)
    \item Dans le cas des chaines finies, $PCTL_{0/1}\subset 
    CTL$!
\end{itemize}
Exemples simples: \[
    P_{>0}(a\mu b)=Ea\mu b\]
    \[P_{>0}X(a)=EXa\]
    \[P_{=1}X(a)=AXa
\]
Par contre
\[
    P_{=1}(a\mu b)=?
\]
ca dépend. Cas général(CM infinies):
\begin{enumerate}
    \item Il n'existe pas d'équivalent CTL à $P_{=1}F(a)$
    ($P_{>0}(Ga)$).\\
On peut regarder $\N$ ou $x_n=p*x_{n+1}+(1-p)x_{n-1}$ 
et $x_0=(1-p)x_0+p*x_1$. La structure de Kripke (CTL)
sous jacente ne depend pas de $p$ et si $p<1/2,>1/2$ 
on a un comportement différent: 
\[
    P_{=1}F0
\]
    \item Il n'existe pas d'équivalent $PCTL_{0/1}$ 
    à $AFa,EGa$. 
\end{enumerate}

Dans les chaines de Markov finies, on a:
\[
    P_{=1}A(EFa)Wa
    \]
Où $W=Weak~until$, s'exprime dans CTL (simplement 
on a pas $a$ sur tout le chemin comme $a\mu b$ je crois).

\subsection{bisimulation, syst.classiques}
\begin{itemize}
    \item 2 structures de Kripke: $K_1=(S_1, \to_1, s_0^1, l_1)$
    $K_2=(S_2, \to_2, s_0^2, l_2)$.
    \item Une relation $R\subset S_1\times S_2$ est une 
    bisimulation ssi $\forall(s_1,s_2)\in R$:
    \begin{enumerate}
        \item $l_1(s_1)=l_2(s_2)$
        \item $\forall s_1\to_1 s_1'$, $\exists s_2\to_2 s_2'$ t.q
        $(s_1',s_2')\in R$
        \item $\forall s_2\to_2 s_2'$, $\exists s_1\to_1 s_1'$ t.q
        $(s_1',s_2')\in R$
    \end{enumerate}
\end{itemize}

Mauvaise def:
\begin{defn}[$M_1=(S_1, P_1, s_0^1,l_1)$,$M_2=\ldots$]
    Une bisimulation probabiliste sur $M_1\times M_2$ est 
    une relation d'équivalence $R\subseteq S_1\times S_2$
    t.q pour tout $(s_1,s_2)\in R$, on a: 
    \begin{enumerate}
        \item $l_1(s_1)=l_2(s_2)$
        \item $P_1(s_1, T)=P_2(s_2, T)$
    \end{enumerate}
\end{defn}
Bonne def:
\begin{defn}[$M=(S, P, s_0,l)$]
    Une bisimulation probabiliste sur $M$ est 
    une relation d'équivalence $R\subseteq S\times S$
    t.q pour tout $(s_1,s_2)\in R$, on a: 
    \begin{enumerate}
        \item $l(s_1)=l(s_2)$
        \item $P(s_1, T)=P(s_2, T)$
    \end{enumerate}
\end{defn}
\begin{thm}
    $s_1\sim s_2\equiv s_1~et~s_2$ vérifient les mêmes
    formules de PCTL!    
\end{thm}
(Chaine quotient ! on prend $\overline{P}=\#classe*P$)


\newpage
\section{Processus de décision de Markov(MDP)}
On a maintenant et des choix non déterministes et des probas.
(Mélanges des deux d'avants).\\

\begin{defn}
    Un MDP, $M=(S,s_0, Act, Pr, l)$:
    \begin{itemize}
        \item $Act=$ ensemble fini d'actions
        \item $Pr:S\times Act\times S\to [0,1]$ tq 
        $\forall s\forall \alpha\in Act$, 
        \[\sum_{t\in S}\Pr(s,\alpha, t)\in\{0,1\}\]
    \end{itemize}
\end{defn}

\textbf{Notation:}
\begin{itemize}
    \item Une action $\alpha$ est "tirable" ou 
possible depuis un état $s$ ssi $\sum_{t\in S}Pr(s,\alpha,t)=1$
    \item $Act(s)=$ actions possibles depuis $s$
    \item $Supp(M)=\{s|Act(s)\ne\emptyset\}$
    \item Si $Pr(s, \alpha, t)>0$ on dit que 
$t$ est un $\alpha$-successeur de $s$.
\end{itemize}

Comme pour les chaines de Markov, on définit la
structure de Kripke $K_M$ avec $K_M=(S,s_0,\to, l)$ 
ou cette fois $s\to s'$ ssi $\exists \alpha\in Act(s)$
tq $Pr(S,\alpha, s')>0$. De même on définit les chemins 
avec les actions.


($S^+$ l'ensemble des séquences finies de $S$)
(Ici le Scheduler se souvient de la ou il passe pour déterminer 
les actions d'après)
\begin{defn}
    Un scheduler pour $M$ est une fct $\sigma:S^+\to Act$
    tq $\sigma(s_0\ldots s_n)\to Act(s_n)$
\end{defn}

\begin{prop}
    La donnée d'un MDP et d'un scheduler est une chaine de 
    Markov! $M_{\sigma}=(S', s_0, P, l')$:
    \begin{itemize}
        \item $S'=S^+$
        \item $P(s', t)=Pr(s_n,\sigma(s'), t)$
    \end{itemize}
\end{prop}

-Les Scheduler sans mémoire: 
\begin{defn}
    $\sigma:S\to Act$, 
\end{defn}

-Les Scheduler à mémoire finie:
\begin{defn}
    $\sigma=(Q, q_0, \Delta, act):$
    \begin{itemize}
        \item $\Delta: Q\times S\to Q$
        \item $act:Q\times S\to Act$ tq 
        \[act(q,s)=\text{Donne l'action à jouer depuis l'état s
        de la MDP lorsque la  mémoire est dans le mode q}\]
        \[\Delta(q,s)=\text{Le prochain état de la mémoire}\]
    \end{itemize}
\end{defn}

-Les Scheduler randomisés: Cette fois on va de $S^+->Dist(Act)$
On associe des probabilités d'avoir une action


\subsection{}
Approche alternative pour l'analyse des MDP.
\begin{itemize}
    \item Objectif: obtenir un algo qui permet d'avoir 
    un intervalle de proba.
    \item Idée: On va remplacer le MDP $M$ par un MDP 
    $M^{min}$ (ou $M^{max}$) ayant une structure particuliere 
    qui permet le calcul des intervalles de probas (et qui 
    conservent les probas min/max).
\end{itemize}

Décomposition d'un MDP en MEC (Maximal End Component).
\begin{defn}[End Component]
    $M=(S,s_0,Act, P, l)$ une MDP,
    \begin{enumerate}
        \item une sous-MDP de $M$ est défini par un $S'
        \subseteq S$ et $Act'$ t.q $S'\ne \emptyset$
        ($\forall s\in S',~Act'(s)\subseteq Act(s)$)
        \item $(S',Act')$ est un End Component ssi 
        $\forall s\in S'$, $\forall \alpha\in Act'(s)$
        on a $\forall t\in S, P(s,\alpha,t)>0\implies t\in S'$.
    \end{enumerate}
\end{defn}

\begin{itemize}
    \item On autorise les EC de la forme $(\{s\}, \emptyset)$.
    \item Etant données deux EC $(S_1, Act_1)$, $(S_2, Act_2)$,
    d'une MDP $M$:
    \begin{enumerate}
        \item $(S_1, Act_1)\leq (S_2, Act_2)$ si $S_1\subseteq S_2$
        et $\forall s\in S_1$, $Act_1(s)\subseteq Act_2(s)$
        \item Si $S_1\cap S_2\ne \emptyset$, alors 
        \[(S_1\cup S_2, Act_1\cup Act_2)\]
        est un EC. (y'a de la transitivité)
    \end{enumerate}
    \item On a une notion de Maximal EC (1.).
    \item deux MEC ne partagent pas d'état $(2.)$. Cela 
    définit une partition de l'ensemble $S$.
\end{itemize}

On distingue $3$ types de MEC:
\begin{enumerate}
    \item Les TMEC: Trivial MEC$\to$$(\{s\}, \emptyset)$.
    \item Les BMEC: BottomMEC/TerminalMEC$\to$$(S',Act')$
    t.q $Act'$ est maximal (donne même actions que $Act$)
    \item Les autres.
\end{enumerate}

%https://q.uiver.app/#q=WzAsNyxbNCw1LCIodSkiXSxbNiw1LCIodikiXSxbMCwzLCIocykiXSxbMiwzLCIocycpIl0sWzQsMSwiKHMrKSJdLFs0LDMsIih0KSJdLFs1LDAsIihzKykiXSxbMSwwLCJcXGFscGhhIiwwLHsiY3VydmUiOjN9XSxbMCwxLCJcXGFscGhhIiwwLHsiY3VydmUiOjN9XSxbNSw0LCIwLjIsXFxhbHBoYSJdLFs1LDMsIjAuNSxcXGFscGhhIiwyXSxbNSwwLCIwLjMsXFxhbHBoYSIsMV0sWzMsMCwiMC41LFxcYmV0YSIsMSx7ImN1cnZlIjoyfV0sWzMsNCwiMC41LFxcYmV0YSIsMV0sWzIsMywiXFxhbHBoYSIsMix7ImN1cnZlIjoyfV0sWzMsMiwiXFxhbHBoYSIsMix7ImN1cnZlIjoyfV0sWzQsNiwibG9vcCIsMix7ImN1cnZlIjozfV0sWzYsNCwibG9vcCIsMix7ImN1cnZlIjo0fV1d
% \[\begin{tikzcd}
% 	&&&&& {(s+)} \\
% 	&&&& {(s+)} \\
% 	\\
% 	{(s)} && {(s')} && {(t)} \\
% 	\\
% 	&&&& {(u)} && {(v)}
% 	\arrow["\alpha", curve={height=18pt}, from=6-7, to=6-5]
% 	\arrow["\alpha", curve={height=18pt}, from=6-5, to=6-7]
% 	\arrow["{0.2,\alpha}", from=4-5, to=2-5]
% 	\arrow["{0.5,\alpha}"', from=4-5, to=4-3]
% 	\arrow["{0.3,\alpha}"{description}, from=4-5, to=6-5]
% 	\arrow["{0.5,\beta}"{description}, curve={height=12pt}, from=4-3, to=6-5]
% 	\arrow["{0.5,\beta}"{description}, from=4-3, to=2-5]
% 	\arrow["\alpha"', curve={height=12pt}, from=4-1, to=4-3]
% 	\arrow["\alpha"', curve={height=12pt}, from=4-3, to=4-1]
% 	\arrow["loop"', curve={height=18pt}, from=2-5, to=1-6]
% 	\arrow["loop"', curve={height=24pt}, from=1-6, to=2-5]
% \end{tikzcd}\]

Le problème qu'on veut: $Fs$? On le résout depuis chaque MEC facilement.
On a une partition de $S$ données par les $BMEC,TMEC,autres$(Leurs états).

\textbf{Algorithme pour déterminer les MEC de $M$:}
\begin{itemize}
    \item input: $M$.
    \item output: ensemble des MEC de $M$.
\end{itemize}
On fait :
\begin{itemize}
    \item $P=Pile$
    \item $P.push((S, Act))$ (init avec M)
    \item $S_{MEC}=\emptyset$
    \item Tant que $P\ne \emptyset$:
    \begin{itemize}
        \item $(S', Act'):= P.pop()$
        \item Pour $s\in S'$ et $\alpha\in Act'(s)$:
        \begin{itemize}
            \item Pour $t\in S$ t.q. $P(s,\alpha, t)>0$:
            \item si $t\notin S'$ alors $Act'(s)=Act'(s)\backslash\{\alpha\}$.
        \end{itemize}
        \item Calculer les $CFC$ du graphe $(S', Act')\to (S_1,\ldots, S_k)$.
        \item Si $k>1$ alors:
        \begin{itemize}
            \item Pour $i=1$ à $k$:
            \item $P.push((S_i, Act'))$
        \end{itemize}
        \item Sinon: $SMEC+=\{(S',Act')\}$
    \end{itemize}
    \item Retourne $SMEC$
\end{itemize}

Pour la MDP de quiver:
\begin{itemize}
    \item $(S,Act)$
    \item nettoyage de $Act$, ok.
    \item $CFC$ du graphe $\{s,s'\}$, $\{u,v\}, \{s+\}, \{t\}$
    \item $P:~(\{s,s'\},Act), (\{u,v\},Act), (\{s+\}, Act), (\{t\},Act)$
    \item $(\{s,s'\}, Act)\to$ supprimer $\beta$ depuis $S'$, $\to$ MEC$=(\{s,s'\},Act')$
\end{itemize}

\subsection{Décomposition pour MIN}
\textbf{"Min-Réduction"}\\

(Apparemment probabilité nulle sur les états des 
BMEC, S\_k, ducoup on les fusionne en $s-$)
\begin{defn}
    $M=(S,s_0,Act, P,l)$ et $S=\bigcup_{m=0}^{M}B_m\cup\bigcup_{l=1}^{L}\{t_l\}\cup\bigcup_{k=1}^{K}S_k$:
    \[\tilde{S}=\{S+, t_1,\ldots,t_L,s-\}\]
    \[\tilde{Act}(s+)=\{loop\}=\tilde{Act}(s-)\]
    \[\tilde{Act}(t_i)=Act(t_i)\]
    Pour les probas:
    \[\tilde{P}(s+, loop, s+)=1=\tilde{P}(s-,loop, s-)\]
    \[\forall\alpha\in Act(t_i):~\tilde{P}(t_i, \alpha, t_j)=P(t_i,loop, t_j)\]
    \[\forall\alpha\in Act(t_i):~\tilde{P}(t_i, \alpha, s+)=P(t_i,loop, s+)\]
    \[\tilde{P}(t_i, \alpha, s_i)=\sum_{k=1}^{K}P(t_i,\alpha, s_k)+\sum_{m=1}^{M}P(t_i, \alpha, B_m)\]
\end{defn}
(Rappel: $P(t,\alpha, U)=\sum_{u\in U} P(t,\alpha, u)$)
C'est la décomposition MIN de celle de tout à l'heure.
% https://q.uiver.app/#q=WzAsNSxbMCw1LCIocy0pIl0sWzAsMSwiKHMrKSJdLFswLDMsIih0KSJdLFsxLDAsIihzKykiXSxbMiw1LCIocy0pIl0sWzIsMSwiMC4yLFxcYWxwaGEiXSxbMiwwLCIwLjUrMCswLjMrMD0wLjgsXFxhbHBoYSIsMV0sWzEsMywibG9vcCwgMSIsMix7ImN1cnZlIjozfV0sWzMsMSwibG9vcCwgMSIsMix7ImN1cnZlIjo0fV0sWzQsMCwibG9vcCwxIiwwLHsiY3VydmUiOjN9XSxbMCw0LCJsb29wLDEiLDAseyJjdXJ2ZSI6M31dXQ==
% \[\begin{tikzcd}
% 	& {(s+)} \\
% 	{(s+)} \\
% 	\\
% 	{(t)} \\
% 	\\
% 	{(s-)} && {(s-)}
% 	\arrow["{0.2,\alpha}", from=4-1, to=2-1]
% 	\arrow["{0.5+0+0.3+0=0.8,\alpha}"{description}, from=4-1, to=6-1]
% 	\arrow["{loop, 1}"', curve={height=18pt}, from=2-1, to=1-2]
% 	\arrow["{loop, 1}"', curve={height=24pt}, from=1-2, to=2-1]
% 	\arrow["{loop,1}", curve={height=18pt}, from=6-3, to=6-1]
% 	\arrow["{loop,1}", curve={height=18pt}, from=6-1, to=6-3]
% \end{tikzcd}\]

%La semaine dernière y ont montré qu'il existe un scheduler
%qui réalise la proba min
\begin{prop}
    $\forall M, \forall s\in S$, on a 
    \[Pr_M^{min}(s\vDash Fs+)=Pr_{M^{min}}^{min}(\tilde{s}\vDash Fs+)\] 
\end{prop}
\textbf{Preuve:} En gros on a un scheduler $\sigma$ qui réalise
la proba min dans $M$. On en déduit un $\tilde{\sigma}$ qui 
lui correspond. Pour les états fusionnés, la proba min
est 0 avec loop, c'est ok. Pour les autres, on joue la 
meme section dans $M$ et dans $M^{min}$(Le truc c'est que 
c'est inversible).\qed

Dans notre cas: $Pr^{min}(t\vDash Fs+)=0.2$.\\
Un autre exemple: (En commentaire dans le tex)\\
% https://q.uiver.app/#q=WzAsMTAsWzAsMSwic18wIl0sWzIsMSwic18xIl0sWzQsMSwic18yIl0sWzYsMSwicysiXSxbMiwzLCJzXzMiXSxbMCwzLCJzXzUiXSxbNCwzLCJzXzQiXSxbMCwwLCJcXGJ1bGxldCJdLFsyLDAsIlxcYnVsbGV0Il0sWzYsMCwiXFxidWxsZXQiXSxbMCwxLCIxLzIsXFxhbHBoYSIsMV0sWzEsMiwiMS8yIiwxXSxbMiwzLCIzLzQiLDFdLFs2LDMsIjEvMixcXGJldGEiLDEseyJjdXJ2ZSI6Mn1dLFs0LDMsIjIvMyxcXGJldGEiLDFdLFs0LDYsIjEsXFxhbHBoYSIsMV0sWzYsNCwiMSxcXGFscGhhIiwxLHsiY3VydmUiOi0yfV0sWzQsNSwiMS8zLFxcYmV0YSIsMV0sWzAsNCwiMS8yLFxcYmV0YSIsMV0sWzAsNSwiMS8yLFxcYmV0YSIsMV0sWzIsNSwiMS80IiwxLHsiY3VydmUiOi0yfV0sWzYsNSwiMS8yLFxcYmV0YSIsMSx7ImN1cnZlIjotNX1dLFswLDcsIjEvMixcXGFscGhhIiwxXSxbNywwLCIiLDEseyJjdXJ2ZSI6Mn1dLFsxLDgsIjEvMixcXGFscGhhIiwxXSxbOCwxLCIiLDEseyJjdXJ2ZSI6Mn1dLFszLDksImxvb3AiLDFdLFs5LDMsIiIsMSx7ImN1cnZlIjoyfV1d
% \[\begin{tikzcd}
% 	\bullet && \bullet &&&& \bullet \\
% 	{s_0} && {s_1} && {s_2} && {s+} \\
% 	\\
% 	{s_5} && {s_3} && {s_4}
% 	\arrow["{1/2,\alpha}"{description}, from=2-1, to=2-3]
% 	\arrow["{1/2}"{description}, from=2-3, to=2-5]
% 	\arrow["{3/4}"{description}, from=2-5, to=2-7]
% 	\arrow["{1/2,\beta}"{description}, curve={height=12pt}, from=4-5, to=2-7]
% 	\arrow["{2/3,\beta}"{description}, from=4-3, to=2-7]
% 	\arrow["{1,\alpha}"{description}, from=4-3, to=4-5]
% 	\arrow["{1,\alpha}"{description}, curve={height=-12pt}, from=4-5, to=4-3]
% 	\arrow["{1/3,\beta}"{description}, from=4-3, to=4-1]
% 	\arrow["{1/2,\beta}"{description}, from=2-1, to=4-3]
% 	\arrow["{1/2,\beta}"{description}, from=2-1, to=4-1]
% 	\arrow["{1/4}"{description}, curve={height=-12pt}, from=2-5, to=4-1]
% 	\arrow["{1/2,\beta}"{description}, curve={height=-30pt}, from=4-5, to=4-1]
% 	\arrow["{1/2,\alpha}"{description}, from=2-1, to=1-1]
% 	\arrow[curve={height=12pt}, from=1-1, to=2-1]
% 	\arrow["{1/2,\alpha}"{description}, from=2-3, to=1-3]
% 	\arrow[curve={height=12pt}, from=1-3, to=2-3]
% 	\arrow["loop"{description}, from=2-7, to=1-7]
% 	\arrow[curve={height=12pt}, from=1-7, to=2-7]
% \end{tikzcd}\]

Les MEC: $(\{s+\}, loop), (\{s_5\},loop),
 (\{s_1\},\emptyset), (\{s_2\}, \emptyset), (\{s_3, s_4\}, (s_3\to\alpha, s_4\to\alpha))$

% https://q.uiver.app/#q=WzAsOSxbMCwxLCJzXzAiXSxbMiwxLCJzXzEiXSxbNCwxLCJzXzIiXSxbNiwxLCJzKyJdLFswLDAsInNfMCJdLFsyLDAsInNfMSJdLFs2LDAsInMrIl0sWzIsMywicy0iXSxbMiw0LCJzLSJdLFswLDEsIjEvMixcXGFscGhhIiwxXSxbMSwyLCIxLzIsIFxcYWxwaGEiLDFdLFsyLDMsIjMvNCIsMV0sWzAsNCwiMS8yLFxcYWxwaGEiLDFdLFs0LDAsIiIsMSx7ImN1cnZlIjoyfV0sWzEsNSwiMS8yLFxcYWxwaGEiLDFdLFs1LDEsIiIsMSx7ImN1cnZlIjoyfV0sWzMsNiwibG9vcCIsMV0sWzYsMywiIiwxLHsiY3VydmUiOjJ9XSxbMCw3LCIxLFxcYmV0YSIsMV0sWzIsNywiMS80IiwxXSxbNyw4XSxbOCw3LCJsb29wIiwxLHsiY3VydmUiOjJ9XV0=
%  \[\begin{tikzcd}
%      {s_0} && {s_1} &&&& {s+} \\
%      {s_0} && {s_1} && {s_2} && {s+} \\
%      \\
%      && {s-} \\
%      && {s-}
%      \arrow["{1/2,\alpha}"{description}, from=2-1, to=2-3]
%      \arrow["{1/2, \alpha}"{description}, from=2-3, to=2-5]
%      \arrow["{3/4}"{description}, from=2-5, to=2-7]
%      \arrow["{1/2,\alpha}"{description}, from=2-1, to=1-1]
%      \arrow[curve={height=12pt}, from=1-1, to=2-1]
%      \arrow["{1/2,\alpha}"{description}, from=2-3, to=1-3]
%      \arrow[curve={height=12pt}, from=1-3, to=2-3]
%      \arrow["loop"{description}, from=2-7, to=1-7]
%      \arrow[curve={height=12pt}, from=1-7, to=2-7]
%      \arrow["{1,\beta}"{description}, from=2-1, to=4-3]
%      \arrow["{1/4}"{description}, from=2-5, to=4-3]
%      \arrow[from=4-3, to=5-3]
%      \arrow["loop"{description}, curve={height=12pt}, from=5-3, to=4-3]
%  \end{tikzcd}\]
On obtient: 
\[x_0=min(0,(x_0+x_1)/2)\]
\[x_1=(x_1+x_2)/2=3/4\]
\[x_2=3/4\]

\begin{prop}
    Dans $M^{min}$, tout état de $\tilde{S}\backslash\{s+,s-\}$ est 
    une $TMEC$ pour $M^{min}$.(et $s+$ et $s-$ sont des BMEC)
\end{prop}

\begin{prop}
    Dans $M^{min}$, tout chemin ''finit'' avec proba $1$ 
    dans $s-$ ou $s+$. (On peut pas boucler dans les $t$!)
\end{prop}
\textbf{Preuve:} $G_0=\{s-,s+\}$, \[G_{i+1}=
\{s\in \tilde{S}\backslash \bigcup_{j\leq i} G_j|
\forall \alpha\in \tilde{Act}(s), \exists s'\in\bigcup_{j\leq i}G_j
~t.q.~ P(s,\alpha, s')>0\}\]
Les $G_i$ partitionnent $\tilde{S}$. Supp que $s\in \tilde{S}-\bigcup G_i=P$.
Pour tout $s\in P$, il existe $\alpha\in Act(s)$, tout les 
états tq $P(s,\alpha, t)>0$ sont dans $P$. $P$ définit 
une sous MDP de $M$. (Bon la preuve est reloue)

\begin{prop}
    
\end{prop}

\subsection{Décomposition pour MAX}

\begin{defn}
    $M=(S\ldots)$, $S=\bigcup_{m=0}^{M}B_m\cup\bigcup_{l=1}^{L}\{t_l\}\cup\bigcup_{k=1}^{K}S_k$. 
    La "max-réduction" $M^{max}$ de $M$ est un MDP=$\bar{M}$:(On veut garantir l'unicité du point fixe)
    \begin{itemize}
        \item $\bar{S}=\{s-,s+,t_1,\ldots,t_K, s_1,\ldots,s_K\}$
        \item $\bar{Act}(s-)=Act(s+)=\{loop\}$
        \item $\bar{Act}(t_i)$, $\forall 1\leq k\leq K$, 
        \[
            \bar{Act}(s_k)=\{\alpha | \exists s\in S_k\wedge 
            \alpha\in Act(s)\wedge \exists t\in S-S_k~tq~Pr(s,\alpha,t)>0\}
        \]
        \item $\bar{P}(t_i,\alpha, t_j)=P(t_i,\alpha, t_j)$
        \item Pareil que min pour $s+, s-,t_i$
        \item Pour $s\in S_k$ tq $\alpha\in Act(s)$: $\bar{P}(s_k,\alpha, s+)=P(s,\alpha,s+)~pour~s\in S_k~tq~\alpha\in Act(s)$
        \item $\bar{P}(s_k, \alpha, s-)=P(s,\alpha, \bigcup B_i)$
        \item Pour $\alpha\in Act(s)$, $s\in S_k$: $\bar{P}(s_k, \alpha, t_i)=P(s,\alpha, t_i)$
        \item Pour $\alpha\in Act(s)$, $s\in S_k$, $s'\in S_{k'}$: $\bar{P}(s_k, \alpha, s_{k'})$
    \end{itemize}
\end{defn}

Le $M^{max}$ obtenu pour la chaine de tout à l'heure: (en commentaire dans le tex)
% https://q.uiver.app/#q=WzAsNixbMiw1LCIodSkiXSxbNCw1LCIodikiXSxbMCwzLCIocykiXSxbMiwxLCIocyspIl0sWzIsMywiKHQpIl0sWzMsMCwiKHMrKSJdLFsxLDAsImxvb3AiLDAseyJjdXJ2ZSI6M31dLFswLDEsImxvb3AiLDAseyJjdXJ2ZSI6M31dLFs0LDMsIjAuMixcXGFscGhhIl0sWzQsMiwiMC41LFxcYWxwaGEiLDJdLFs0LDAsIjAuMyxcXGFscGhhIiwxXSxbMiwwLCIwLjUsXFxiZXRhIiwxLHsiY3VydmUiOjJ9XSxbMiwzLCIwLjUsXFxiZXRhIiwxXSxbMyw1LCJsb29wIiwyLHsiY3VydmUiOjN9XSxbNSwzLCJsb29wIiwyLHsiY3VydmUiOjR9XSxbMiwzLCIwLjMsXFxhbHBoYSIsMix7ImN1cnZlIjotNX1dLFsyLDAsIjAuNyxcXGFscGhhIiwxLHsiY3VydmUiOjV9XV0=
% \[\begin{tikzcd}
% 	&&& {(s+)} \\
% 	&& {(s+)} \\
% 	\\
% 	{(s)} && {(t)} \\
% 	\\
% 	&& {(u)} && {(v)}
% 	\arrow["loop", curve={height=18pt}, from=6-5, to=6-3]
% 	\arrow["loop", curve={height=18pt}, from=6-3, to=6-5]
% 	\arrow["{0.2,\alpha}", from=4-3, to=2-3]
% 	\arrow["{0.5,\alpha}"', from=4-3, to=4-1]
% 	\arrow["{0.3,\alpha}"{description}, from=4-3, to=6-3]
% 	\arrow["{0.5,\beta}"{description}, curve={height=12pt}, from=4-1, to=6-3]
% 	\arrow["{0.5,\beta}"{description}, from=4-1, to=2-3]
% 	\arrow["loop"', curve={height=18pt}, from=2-3, to=1-4]
% 	\arrow["loop"', curve={height=24pt}, from=1-4, to=2-3]
% 	\arrow["{0.3,\alpha}"', curve={height=-30pt}, from=4-1, to=2-3]
% 	\arrow["{0.7,\alpha}"{description}, curve={height=30pt}, from=4-1, to=6-3]
% \end{tikzcd}\]

\begin{prop}
    \begin{itemize}
        \item Les probas max sont conservées en passant à $M^{max}$.
        \item Dans $M^{max}$, tout état de $\bar{S}-\{s+,s-\}$ sont des TMEC.
    \end{itemize}
    \[
        fmax:V\to V;~fmax(\bar{x})=(max_{\alpha\in Act(s)} \sum_{s\in \bar{S}} Pr(s,\alpha, t).x+)_{s\in \bar{S}}\]
\end{prop}
\begin{prop}
    Le vecteur $P_{M^{max}}^{max}(Fs+)$ est l'unique point 
    fixe de fmax! (pas vrai pour min?)
\end{prop}

\textbf{Algo:}(min)
On définit deux séquences $x$ (sous approx de $P^{min}$) et $y$ (sur approx de $P^{min}$).
$\begin{cases}
    &x^{(0)}=(1~pour~s+,0~sinon)\\
    &y^{(0)}=(0~pour~s-,1~sinon)\\
\end{cases}$
et $\begin{cases}
    &x^{(n+1)}=f_{min}(x^{(n)})\\
    &y^{(0)}=f_{min}(y^{(n)})\\
\end{cases}$.\\
(max): $\begin{cases}
    &x^{(0)}=(1~pour~s+,0~sinon)\\
    &y^{(0)}=(0~pour~s-,1~sinon)\\
\end{cases}$
et $\begin{cases}
    &x^{(n+1)}=f_{max}(x^{(n)})\\
    &y^{(0)}=f_{max}(y^{(n)})\\
\end{cases}$.\\

\begin{prop}
    L'algo converge en au plus $n.[\log(\epsilon)/\log(1-\eta^n)]$. 
    ($n$ le nombres des $G_i$ et $\eta$ la proba min $>0$ dans $M$)
\end{prop}
\end{document}
